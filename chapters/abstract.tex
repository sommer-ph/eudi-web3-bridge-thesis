\section*{Abstract}
This thesis investigates how \acrshort{zksnark}s can be applied to bridge EU regulatory requirements, such as identity binding in financial transactions, with the privacy principles of Web3. The central research question is whether monolithic or recursive proof composition offers better efficiency when binding \acrshort{eudi} credentials to blockchain wallets. At its core, the thesis introduces the credential–wallet binding proof ($\pi_{\text{cred-bind}}$), which enables a user to prove possession of a valid \acrshort{eudi} credential together with control of a blockchain wallet public key. This construction is extended with the derived key binding proof ($\pi_{\text{key-bind}}$), which recursively verifies $\pi_{\text{cred-bind}}$ and establishes the binding for child public keys. Both proofs are designed to satisfy the standard \acrshort{zksnark} properties of completeness, knowledge soundness, and zero-knowledge.

We implement $\pi_{\text{cred-bind}}$ as a monolithic Groth16 proof using Circom, snarkjs, and RapidSnark, and also as a monolithic Plonky2 proof, while $\pi_{\text{key-bind}}$ is realized as a recursive Plonky2 proof. All implementations incorporate targeted optimizations to improve proving efficiency while preserving correctness and security. Our evaluation shows that the Groth16 instantiation achieves the highest efficiency, with proof generation in 8.24 seconds, a proof size of 803 bytes, and verification in 0.01 seconds, outperforming both Plonky2 variants. Within Plonky2, recursion yields clear efficiency gains over its monolithic baseline, reducing proving time from 85.05 to 18.39 seconds and proof size from 179.6 to 163 kB.

A simplified cost model explains these results: Groth16 benefits from small constant factors and constant-size proofs, while Plonky2 incurs higher costs due to domain padding and the overhead of FFTs, Merkle commitments, and FRI queries. Nova was also considered as a folding-based alternative, but full integration was not possible because gadget support for the required curves is still missing.

The results highlight three main contributions of this work. First, the definition of two new \acrshort{zksnark}-based proofs for linking \acrshort{eudi} credentials with blockchain wallets. Second, the implementation of both monolithic and recursive approaches in Groth16 and Plonky2. Third, an empirical evaluation and cost model to clarify the performance trade-offs. Taken together, these findings demonstrate that \acrshort{zksnark}s provide a viable cryptographic bridge between EU-compliant digital identity and decentralized wallets. They show that Groth16 currently offers the most efficient solution, while recursion within Plonky2 improves performance internally but remains less competitive across systems. This establishes both the feasibility and the limitations of privacy-preserving yet regulation-compliant identity integration in Web3.