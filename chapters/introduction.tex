\chapter{Introduction}
\label{chap:introduction}

\section{Motivation and problem statement}
\label{sec:motivation-problem-statement}
% EU context
In recent years, the concept of digital identity has become a central element of the digital transformation in the European Union (EU). To address the fragmentation of national solutions and to ensure secure, interoperable, and privacy-preserving identity management, the EU launched the \textit{\acrfull{eudi}} initiative~\cite{EUCommission2021, EURegulation2024}. At its core is the \textit{\acrshort{eudi} Wallet}, a user-controlled application for storing and presenting legally verified credentials such as identification data, certificates, and electronic signatures. Under Regulation (EU 2024/1183), each Member State must provide a compliant wallet. The \acrshort{eudi} Wallet serves as a cryptographic anchor for legally verified credentials and enables user-controlled identity management across borders and sectors. However, the \acrshort{eudi} framework is closely tied to broader EU regulation, including the \textit{\acrfull{aml}} and the \textit{\acrfull{mica}}~\cite{EUDirective2015AML,EURegulation2023MiCA}. These measures require that users in financial services and regulated value transfers can be linked to verifiable legal identities, typically through \textit{\acrfull{kyc}} procedures and digitally signed credentials. Together, they position the \acrshort{eudi} Wallet as a central element of the EU's digital identity ecosystem. Through trusted credentials and selective disclosure, the wallet provides a standardized basis for privacy-preserving digital identity management across the EU.

% Web3 context
In contrast to this regulatory vision, Web3 represents a fundamentally different paradigm for identity and interaction. It promotes decentralization, user autonomy, pseudonymity, and self-sovereign identity systems \cite{bambacht2022web3}. These principles are reflected in blockchain-based architectures, which deliberately avoid inherent links to users’ real-world identities. A common technical component in such systems is the use of \acrfull{hd} wallets, most notably specified in \acrfull{bip32} \cite{bip32, Das2019, narayanan2016bitcoin}. HD wallets derive child keys from a parent key, which prevents key reuse and preserves unlinkability across multiple interactions, while never embedding legal identity information. This architecture supports privacy and censorship resistance, allowing pseudonymous interaction without relying on trusted intermediaries. At the same time, it conflicts with EU regulatory frameworks that require strong identity binding in financial and legally accountable applications. Directly linking an \acrshort{eudi} credential to a wallet key would break unlinkability, undermining Web3’s privacy guarantees. Moreover, prior work shows that key compromise in \acrshort{hd} wallets can propagate along the hierarchy, potentially exposing large parts of a user’s assets~\cite{Das2019}. Thus, weakening unlinkability not only threatens privacy but also creates concrete security risks such as large-scale key exposure and asset loss.

% Problem statement
Bridging the gap between the requirements of regulatory identity verification and the privacy-preserving design principles of decentralized architectures presents a fundamental design challenge at the intersection of these two paradigms. This challenge arises on two levels. First, users must be able to prove that they possess a valid \acrshort{eudi} credential and control a specific blockchain wallet typically represented by a public key, without revealing any link between their legal identity and the wallet. Second, users must demonstrate that a child key used in a transaction has been correctly derived from a parent key of the wallet, without compromising the unlinkability between individual transactions. In both cases, the goal is to preserve privacy while enabling compliance, without disclosing identifying attributes or wallet internals. This conflict between regulatory identity binding and cryptographic unlinkability can be addressed using a \acrfull{zkp}, which is a group of cryptographic protocols that allows one party (the prover) to convince another party (the verifier) that a statement is true, without revealing any information beyond the validity of the statement itself \cite{narayanan2016bitcoin, 10.1145/22145.22178}. A particularly promising type of \acrshort{zkp}s is the \acrfull{zksnark}. The term refers to proof systems which enable short, non-interactive, and efficiently verifiable proofs for general statements expressed over arithmetic circuits \cite{liang2025}. Importantly, they offer strong privacy guarantees, making them ideal for regulatory-compliant yet privacy-preserving blockchain use cases \cite{rosenberg2023, Baldimtsi_2024}. At a methodological level, \acrshort{zksnark}s can be realized either by encoding all required constraints in a single circuit, or by composing multiple proofs in a recursive manner. Which of these two composition strategies yields better efficiency in the specific context of binding EU-compliant credentials to blockchain wallets constitutes the central question of this thesis.

\section{Contribution and structure}
\label{sec:contribution-structure}
This thesis addresses the challenge of bridging the gap between regulatory identity compliance and privacy-preserving identity management in decentralized systems. The main contribution consists of the design and implementation of zero-knowledge circuits for the following statements:

The first is the \emph{credential–wallet binding}, which enables a user to prove possession of a valid \acrshort{eudi} credential while demonstrating control over a blockchain wallet represented by a public key. Both the credential and the wallet’s secret key remain private witness inputs to the \acrshort{zksnark}, ensuring that no direct link between them is revealed. The second is the \emph{derived key binding}, which proves that a child public key used in a transaction has been correctly derived from a parent public key of the wallet. Here, the child key is public, whereas the parent key remains part of the private witness, thus preserving unlinkability between individual transactions.

\medskip
The two statements are implemented using distinct composition approaches. The credential–wallet binding proof is realized as a monolithic circuit that encapsulates all required constraints in a single proof. The derived key binding proof, in contrast, follows a recursive composition, in which the circuit not only enforces correct key derivation but also verifies a previously generated credential–wallet binding proof. This separation supports flexible deployment and enables a structured evaluation of performance trade-offs between the two approaches.
\begin{quote}
	\textbf{Research question:} 
	\textit{Which composition approach, monolithic or recursive, offers better efficiency in terms of proof size, generation time, and verification performance when binding EU-compliant credentials to blockchain wallet operations using \acrshort{zksnark}s?}
\end{quote}
To address this question, the thesis implements and optimizes both approaches and compares their practical performance to assess their suitability for real-world applications. The structure of the thesis is as follows. Chapter~\ref{chap:preliminaries} introduces the foundational concepts and technologies used throughout this work, including wallets, hierarchical key derivation schemes, and \acrshort{zksnark}s. Chapter~\ref{chap:construction} defines the system model with roles, assumptions, and requirements, and then presents the construction of the \acrshort{zksnark}-based proof systems. Chapter~\ref{chap:implementation-evaluation} describes the realization of both proofs using monolithic and recursive composition, together with selected design decisions and implementation-specific optimizations. It further reports the empirical evaluation of proof size, generation time, and verification time, and discusses the implications of the results for real-world deployments. Finally, Chapter~\ref{chap:conclusion} summarizes the findings of the thesis and outlines directions for future work.

\section{Related work}
\label{sec:related-work}
Recent advances in \acrshort{zkp}s have sparked interest in applying \acrshort{zksnark}s to identity-related use cases in decentralized environments. Prominent examples include \textit{zkLogin} and \textit{zkCreds}, which demonstrate how existing identity credentials can be combined with \acrshort{zksnark}s to enable privacy-preserving authentication and credential systems~\cite{Baldimtsi_2024,rosenberg2023}. \textit{zkLogin} leverages \acrfull{oidc} tokens from traditional identity providers to enable pseudonymous login to blockchain applications using \acrshort{zksnark}s. The authors identify latency as a major challenge in mobile scenarios and propose hybrid delegation schemes and proof caching to reduce user-perceived delay. In contrast, \textit{zkCreds} constructs anonymous credentials from existing identity documents using general-purpose \acrshort{zksnark}s, proving in zero knowledge that a committed credential matches signed legacy data (e.g., e-passport), with issuance recorded via a Merkle-based bulletin board. Their system faces efficiency bottlenecks due to the complexity of verifying Merkle tree inclusion proofs and advanced circuit features such as linkable show proofs and clone resistance, which increase proving costs. Beyond such blockchain-oriented systems, other work has focused on credential schemes with a stronger emphasis on selective disclosure and compatibility with existing infrastructures. Frigo and Shelat propose an anonymous credential scheme leveraging the \acrfull{ecdsa}, and highlight that widely used curves such as \textit{secp256r1} are not optimized for \acrshort{zksnark} circuits due to their non-native arithmetic. Instead of relying on ad-hoc gadgets, they construct a dedicated sum-check and Ligero-based proof system tailored for \textit{secp256r1}, and demonstrate that verification can be performed within tens of milliseconds even on mobile devices~\cite{cryptoeprint:2024/2010}. Paquin et al. introduce \textit{Crescent}, an anonymous credential system that enables selective disclosure of existing credentials such as mobile driver’s licenses. Their construction relies on \acrshort{zksnark}-based verification of \acrshort{ecdsa}-signed credentials, thereby providing privacy guarantees such as unlinkability without requiring the involvement of new issuing authorities or fundamental changes to existing credential infrastructures~\cite{cryptoeprint:2024/2013}. Other initiatives such as \textit{zkKYC} and Sun et al. investigate zero-knowledge-based approaches for \acrshort{kyc} compliance in decentralized finance (DeFi) and general public blockchain environments~\cite{cryptoeprint:2022/321,su142114584}. Rathee et al. identify the cost of on-chain credential verification with \acrshort{zksnark}s as a major scalability constraint and propose batching strategies and audit-token reuse to reduce per-user overhead~\cite{cryptoeprint:2022/1286}. Furthermore, Biedermann et al. present a systematization of efforts to connect \acrshort{eudi} infrastructure with decentralized Web3 technologies~\cite{Biedermann_2024}. Their work outlines conceptual and technical challenges but remains theoretical and does not provide a concrete implementation. Buterin et al. emphasize the need for privacy-preserving compliance mechanisms in blockchain systems and highlight the role of \acrshort{zkp}s as a central tool in achieving this balance~\cite{BUTERIN2024100176}.

While prior approaches demonstrate the potential of \acrshort{zkp}s for privacy-preserving identity verification, they remain limited in scope. Blockchain-oriented systems such as \textit{zkLogin}, \textit{zkCreds}, or \textit{zkKYC} do not address EU-compliant credentials, whereas credential-focused work such as \textit{Crescent} or the scheme by Frigo and Shelat does not provide a concrete mechanism to bind \acrshort{eudi} credentials to blockchain wallets. In particular, no available implementation guarantees unlinkability both between a user’s credential and the parent key of a blockchain wallet and among the child keys derived from that parent key.

This thesis addresses this gap by designing a \acrshort{zksnark}-based proof system that combines credential–wallet binding with derived key binding. The system enables a user to prove possession of a valid \acrshort{eudi} credential and control over a blockchain wallet, without revealing either element or creating a direct link between them. It further allows the user to prove that a child key was correctly derived from a parent key of the wallet while maintaining unlinkability. By implementing the system both as a monolithic circuit and in a recursive composition, the thesis examines whether recursion yields efficiency gains in proof size, generation time, and verification performance.
