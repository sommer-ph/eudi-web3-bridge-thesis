\chapter{Implementation and evaluation}
\label{chap:implementation-evaluation}
In this chapter, we present the practical realization of the construction introduced in Chapter~\ref{chap:construction} and provide an empirical evaluation of its performance. While the preceding chapter established the theoretical soundness of the approach, the objective here is to assess its feasibility when instantiated in concrete zero-knowledge proof systems and to analyze the efficiency implications of specific design choices. The chapter therefore bridges the gap between abstract protocol descriptions and executable systems, turning formal statements into measurable artifacts. In doing so, it directly addresses the central research question of this thesis, namely whether a monolithic or a recursive composition offers better efficiency in practice.

Throughout the chapter, we emphasize three aspects. First, we identify SNARK-friendly\footnote{We use the term \emph{SNARK-friendly} to refer to primitives whose algebraic structure minimizes the number of constraints when mapped into arithmetic circuits, such as Poseidon hashes or elliptic curves with efficient modular arithmetic.} primitives and encodings that reduce the arithmetic overhead of implementing cryptographic sub-protocols, such as elliptic-curve verification. Second, we highlight optimization techniques that substantially improve efficiency. Third, we provide a systematic evaluation of proof size, proving time, and verification time, thereby quantifying the trade-offs between monolithic and recursive instantiations. In addition, we complement these empirical measurements with simplified cost models that highlight the dominant factors in each proof system and allow us to analyze under which conditions recursion becomes more efficient than a monolithic construction. This situates our evaluation within a broader landscape of proof-system trade-offs.

The complete implementation is publicly available on GitHub\footnote{\url{https://github.com/sommer-ph/eudi-web3-bridge}} and comprises approximately \emph{12000} lines of code. It spans multiple languages and toolchains. The backend is written in Java, the monolithic prover is implemented in Circom (compiled to Groth16), and the recursive provers (Plonky2 and Nova) are realized in Rust. While the primary implementation resides in this repository, some optimizations required modifications to external libraries (e.g., non-native arithmetic support in Circom and Plonky2). These were integrated into forked versions of the respective libraries and are referenced in the relevant sections of this chapter.

\section{System architecture}
\label{sec:architecture}
At a high level, the implementation is structured as a pipeline with two layers. The \emph{backend} handles wallet and key management, and prepares statement inputs, while the \emph{proving backends} instantiate these statements in concrete zero-knowledge proof systems. This modular decomposition ensures that the backend exposes a uniform producer–consumer interface to the proving layer. Each prover consumes the backend’s preprocessed variables, partitions them into public inputs and private witnesses according to the circuit definition, and produces a proof~$\pi$. Verification is then carried out by the verifier of the respective framework, which outputs an accept/reject bit. The pipeline as a whole comprises three main stages. First, a \emph{preprocessing phase}, which converts structured credential and wallet data into canonical low-level encodings suitable for the target proof system. Second, the instantiation of different proving backends, ranging from monolithic Groth16 circuits to recursive SNARK frameworks such as Plonky2 and Nova. Third, the corresponding verification routines, which are dictated by the chosen backend.

\paragraph{Components and roles}
We now outline the main components of the implementation and their roles in the pipeline:

\medskip
\textbf{Backend (wallet and key management):} A dedicated component responsible for generating and managing wallet and key material, assembling statement inputs (credential data and wallet keys), and exposing them to the proving backends in a framework-agnostic format. Conceptually, the backend serves as the sole producer of inputs consumed by all prover backends.
	
\medskip
\textbf{Monolithic prover (Groth16):} This component realizes the \emph{credential–wallet binding} as a single Groth16 circuit. The circuit jointly enforces credential validity (with respect to the issuer’s public key $\mathit{pk}_I$ and wallet ownership (with respect to the user’s wallet public key $\mathit{pk}_0$), thereby producing a proof $\pi_{\text{cred-bind}}$ with public inputs $(pk_I, pk_0)$. It serves as the baseline for evaluating the efficiency of a non-recursive instantiation. To contextualize the performance results, we also conducted microbenchmarks of individual sub-constraints, including comparisons between native and non-native arithmetic. These measurements provide a finer-grained view of where the main performance bottlenecks arise in the monolithic design.

\medskip
\textbf{Recursive prover (Plonky2):} The Plonky2 backend realizes a recursive composition with two proof layers. In the first layer, an \emph{inner circuit} enforces the credential–wallet binding constraints and produces a proof $\pi_{\text{cred-bind}}$ with public inputs $(pk_I, pk_0)$. In the second layer, an \emph{outer circuit} verifies $\pi_{\text{cred-bind}}$ and, in addition, enforces the derived key binding, i.e., that a child key $\mathit{pk}_i$ is correctly derived from the parent key $\mathit{pk}_0$. The resulting recursive proof, denoted $\pi_{\text{key-bind}}$, has public inputs $(pk_I, pk_i, cc_0, i)$ and thus attests, in a single argument, both the validity of the credential–wallet binding and the correctness of the derived key relation. Beyond this two-layer composition, the backend supports two additional recursion modes for analysis purposes. First, \emph{serial recursion} distributes the individual constraints of the credential–wallet binding across separate circuits. Each circuit recursively verifies the proof of the previous one, forming a linear chain. While not required for functionality, this fine-grained chaining enables a systematic study of performance trade-offs when recursion is applied at the level of individual constraints. Second, \emph{parallel recursion} generates independent proofs for each credential–wallet binding constraint. These proofs are then jointly verified in the derived key binding circuit, which consolidates them into a single recursive argument. This mode allows analyzing the overhead of combining multiple parallel subproofs into one higher-level proof.

\medskip
\textbf{Recursive prover (Nova):} Nova is included as additional backend to provide a comparative perspective on recursion mechanisms beyond Plonky2. It was integrated via Nova Scotia as a proof-of-concept for expressing the credential–wallet binding relation in a folding-based recursion framework. This integration highlights the portability of the construction across recursion paradigms, even though Nova was not pursued as a primary backend due to performance and feature limitations.

\paragraph{Hardware}
All experiments were executed on a Dell Latitude 5420 running Windows 11 Pro (version 23H2). The machine is equipped with an 11th Gen Intel Core i5-1145G7 CPU (4 cores, 8 threads, base 2.60\,GHz) and 16\,GB RAM. All proving and verification ran inside an Ubuntu environment under WSL2\footnote{Windows Subsystem for Linux: \url{https://learn.microsoft.com/en-us/windows/wsl/}}. The WSL resource limits were set via \texttt{.wslconfig} to \texttt{memory=10GB}, \texttt{processors=6}, and \texttt{swap=12GB}, so the Linux subsystem had access to at most 6 vCores and 10\,GB RAM. No GPU acceleration was used.

\section{Preprocessing}
\label{sec:preprocessing}
Preprocessing serves to reduce the complexity of the circuits by offloading data handling, such as parsing, serialization, and format conversion, to the backend before entering the proving system, thereby not only simplifying circuit structure but also reducing the overall constraint count and improving performance. Implementing complete data parsing within zero-knowledge circuits, especially for structured formats such as JSON or credential and wallet schemas, results in arithmetic overhead. The reason is that parsing requires branching on different cases and handling flexible data layouts, which translate into additional constraints when expressed as an arithmetic circuit. Previous systems also report the impracticality of in-circuit parsing. For example, zkLogin notes that “fully parsing the JSON inside a ZK circuit [...] is likely to be very inefficient”. Similarly, they highlight that “\acrshort{jwt} parsing [...] takes approximately 235k constraints [...]”~\cite{Baldimtsi_2024}. To address this, our implementation also performs parsing outside of the circuits. The backend extracts only the relevant fields (e.g., signature components) and formats them into succinct, system-specific canonical representations. These serve as the public inputs and private witnesses in the proofs, thereby shifting the majority of parsing outside the circuit and leaving only minimal structural validation inside. Preprocessing can thus be regarded as an optimization of the implementation, since it eliminates in-circuit overhead while preserving functional fidelity.

\paragraph{Canonical representations}
Building on the notion of multi-precision arithmetic and canonicalization maps introduced in Section~\ref{sec:zksnarks}, we specify their concrete instantiation for the preprocessing phase of our implementation. Let $\mathbb{F}=\mathbb{F}_q$ denote the scalar field of the target proof system, and fix a limb size $\ell$ with $2^{\ell}<q$. For each target system $\mathsf{sys}\in\{\mathsf{G16},\mathsf{P2}\}$, where $\mathsf{G16}$ denotes Groth16 and $\mathsf{P2}$ denotes Plonky2, we define canonicalization maps. In our implementation, \emph{Groth16} is instantiated over the BN254 pairing-friendly curve and its scalar field~\cite{cryptoeprint:2005/133,circomdocs}. Canonicalization is realized through byte-to-field decomposition. A byte string is split into base-$2^\ell$ limbs $L_\ell:\{0,1\}^* \to \mathbb{F}^t$, where each fragment is embedded as a field element and $t$ depends on the input length. Elliptic-curve points are serialized to bytes and passed through $L_\ell$, which we denote as $E_{\mathrm{pt}}^{(\mathsf{G16})}$, while scalar values such as \acrshort{ecdsa} components are serialized in big-endian form, padded to a fixed width, and likewise mapped via $L_\ell$, denoted $E_{\mathrm{sc}}^{(\mathsf{G16})}$. By contrast, \emph{Plonky2} operates over the Goldilocks field and employs hexadecimal encodings rather than field-native limbs~\cite{Plonky2Draft2022}. A byte string is mapped to its canonical \texttt{0x}-prefixed hex representation $H:\{0,1\}^* \to \{\text{hex strings}\}$. Elliptic-curve points are serialized to affine coordinates $(x,y)$ and each component is mapped through $H$, denoted $E_{\mathrm{pt}}^{(\mathsf{P2})}$, while integers are serialized to fixed-width byte strings and likewise mapped by $H$, denoted $E_{\mathrm{sc}}^{(\mathsf{P2})}$. In addition, we define the identity map $E_{\mathrm{id}}^{(\mathsf{sys})}:\mathbb{Z}^t \to \mathcal{E}_{\mathsf{sys}}^t$, which embeds small integers and vectors thereof (e.g., ASCII code points of structured strings) directly into the target domain. Since all such values lie within the range of the scalar field $\mathbb{F}_q$, they can be embedded without further encoding. The corresponding range constraints are enforced inside the circuit to ensure validity. Together, these canonicalization maps ensure that both cryptographic elements and structural metadata are represented in a form compatible with the respective proving systems, while preserving consistency across different toolchains.

\paragraph{Input domain}
To clarify what kind of structured data the system must handle, we define the input domain $D$ as $D = (\mathcal{W}{\mathrm{eid}}, \mathcal{C}, \mathcal{W}{\mathrm{bc}}, \mathrm{params})$. Here, $\mathcal{W}{\mathrm{eid}}$ denotes the \acrshort{eudi} wallet containing the credential key pair, $\mathcal{C}$ is the \acrshort{eudi} credential itself, $\mathcal{W}{\mathrm{bc}}$ represents the blockchain wallet together with its key material, and $\mathrm{params}$ collects the public system parameters (e.g., issuer public key, curve identifiers, domain separators, endianness conventions, fixed widths). This domain $D$ thus serves as the unified container of all cryptographic material and parameters that form the basis of preprocessing.

\begin{definition}[Preprocessing]
	A deterministic algorithm, which takes as input, for a target system $\mathsf{sys}\in\{\mathsf{G16},\mathsf{P2}\}$, structured data $D$ and outputs a collection of canonicalized variables $V^{(\mathsf{sys})}$.
	\begin{itemize}
		\item \textbf{Input:} Structured data $D = (\mathcal{W}_{\mathrm{eid}}, \mathcal{C}, \mathcal{W}_{\mathrm{bc}}, \mathrm{params})$.
		
		\item \textbf{Output:} $V^{(\mathsf{sys})} \subseteq \mathcal{E}_{\mathsf{sys}}$, where $\mathcal{E}_{\mathsf{G16}}=\mathbb{F}_{\mathrm{BN254}}$ and $\mathcal{E}_{\mathsf{P2}}=\{\text{hex-encoded strings}\} \cup \mathbb{Z}_{<q}$.
		
		\item \textbf{Computation:} $\mathsf{PreProc}$ extracts all attributes required by the proof pipeline from $D$ and encodes them via the canonicalization maps $(L_\ell, E_{\mathrm{pt}}^{(\mathsf{sys})}, E_{\mathrm{sc}}^{(\mathsf{sys})}, H, E_{\mathrm{id}}^{(\mathsf{sys})})$ into elements of $\mathcal{E}_{\mathsf{sys}}$. 
	\end{itemize}
		
	The computation is structured as follows:
	\begin{description}[leftmargin=0cm]				
		\item[$\mathcal{W}_{\mathrm{eid}}$:] Provides the key pair $(sk_c, pk_c)$ associated with the \acrshort{eudi} credential.
						
		\item[$\mathcal{C}$:] Through a dedicated parsing routine, the credential provides the \acrshort{ecdsa} signature components $(r,s)$ as well as the base64url-encoded header and payload, represented as arrays of ASCII code points denoted $\mathit{headerB64}$ and $\mathit{payloadB64}$, together with their respective lengths $\mathit{headerB64Length}$ and $\mathit{payloadB64Length}$. In addition, the parser outputs, for each coordinate $u \in \{x,y\}$, the slicing parameters $(\mathit{off}_u^{\mathrm{B64}}, \mathit{len}_u^{\mathrm{B64}}, \mathit{drop}_u, \mathit{len}_u^{\mathrm{inner}})$, which determine how the payload is segmented so that elliptic-curve coordinates can be reconstructed efficiently inside the circuit. The header and payload arrays, together with their lengths, allow the circuit to recompute the signing input for \acrshort{jws}, thereby proving knowledge of the preimage. Using the payload array in combination with the slicing parameters, the circuit can further decode the public key $pk_c$ directly from the credential and compare it against the witness-derived key, ensuring consistency between the credential contents and the secret key material.
		
		\item[$\mathcal{W}_{\mathrm{bc}}$:] Provides the blockchain wallet information $(sk_0, pk_0, cc_0, pk_i, i)$, including the parent key pair, the chain code, and the child public key at index $i$.
		
		\item[$\mathrm{params}$:] Provides public system parameters such as the issuer’s public key, curve identifiers, domain separators, endianness conventions, and fixed widths.
	\end{description}
	\label{def:preproc}
\end{definition}

\paragraph{Instantiation}  
In the following, we give the concrete instantiations of $\mathsf{PreProc}$ for the Groth16- and Plonky2-based systems. For the monolithic Groth16-based implementation, $\mathsf{PreProc}^{(\mathsf{G16})}(D)$ outputs limb-decomposed cryptographic values\footnote{We use $43$-bit limbs for secp256r1 and $64$-bit limbs for secp256k1.} and directly embedded structural values via $E_{\mathrm{id}}^{(\mathsf{G16})}$:
\[
\begin{aligned}
	V^{(\mathsf{G16})} = \{\, &
	E_{\mathrm{pt}}^{(\mathsf{G16})}(pk_I),\;
	E_{\mathrm{pt}}^{(\mathsf{G16})}(pk_0),\;
	E_{\mathrm{sc}}^{(\mathsf{G16})}(r),\;
	E_{\mathrm{sc}}^{(\mathsf{G16})}(s),\;
	E_{\mathrm{sc}}^{(\mathsf{G16})}(sk_c),\;
	E_{\mathrm{sc}}^{(\mathsf{G16})}(sk_0), \\[2pt]
	& E_{\mathrm{id}}^{(\mathsf{G16})}(\mathit{headerB64}),\;
	E_{\mathrm{id}}^{(\mathsf{G16})}(\mathit{headerB64Length}),\;
	E_{\mathrm{id}}^{(\mathsf{G16})}(\mathit{payloadB64}),\;
	E_{\mathrm{id}}^{(\mathsf{G16})}(\mathit{payloadB64Length}), \\[2pt]
	& E_{\mathrm{id}}^{(\mathsf{G16})}(\mathit{off}_x^{\mathrm{B64}}),\;
	E_{\mathrm{id}}^{(\mathsf{G16})}(\mathit{len}_x^{\mathrm{B64}}),\;
	E_{\mathrm{id}}^{(\mathsf{G16})}(\mathit{drop}_x),\;
	E_{\mathrm{id}}^{(\mathsf{G16})}(\mathit{len}_x^{\mathrm{inner}}), \\[2pt]
	& E_{\mathrm{id}}^{(\mathsf{G16})}(\mathit{off}_y^{\mathrm{B64}}),\;
	E_{\mathrm{id}}^{(\mathsf{G16})}(\mathit{len}_y^{\mathrm{B64}}),\;
	E_{\mathrm{id}}^{(\mathsf{G16})}(\mathit{drop}_y),\;
	E_{\mathrm{id}}^{(\mathsf{G16})}(\mathit{len}_y^{\mathrm{inner}})
	\,\} \subseteq \mathcal{E}_{\mathsf{G16}}.
\end{aligned}
\]

For the recursive Plonky2-based system, $\mathsf{PreProc}^{(\mathsf{P2})}(D)$ outputs cryptographic values as hex-encoded strings and structural attributes via $E_{\mathrm{id}}^{(\mathsf{P2})}$. This includes the derivation parameters required for recursion:
\[
\begin{aligned}
	V^{(\mathsf{P2})} = \{\, &
	E_{\mathrm{pt}}^{(\mathsf{P2})}(pk_I),\;
	E_{\mathrm{pt}}^{(\mathsf{P2})}(pk_0),\;
	E_{\mathrm{pt}}^{(\mathsf{P2})}(pk_i),\;
	E_{\mathrm{sc}}^{(\mathsf{P2})}(cc_0),\;
	E_{\mathrm{sc}}^{(\mathsf{P2})}(i), \\[2pt]
	& E_{\mathrm{sc}}^{(\mathsf{P2})}(r),\;
	E_{\mathrm{sc}}^{(\mathsf{P2})}(s),\;
	E_{\mathrm{sc}}^{(\mathsf{P2})}(sk_c),\;
	E_{\mathrm{sc}}^{(\mathsf{P2})}(sk_0), \\[2pt]
	& E_{\mathrm{id}}^{(\mathsf{P2})}(\mathit{headerB64}),\;
	E_{\mathrm{id}}^{(\mathsf{P2})}(\mathit{headerB64Length}),\;
	E_{\mathrm{id}}^{(\mathsf{P2})}(\mathit{payloadB64}),\;
	E_{\mathrm{id}}^{(\mathsf{P2})}(\mathit{payloadB64Length}), \\[2pt]
	& E_{\mathrm{id}}^{(\mathsf{P2})}(\mathit{off}_x^{\mathrm{B64}}),\;
	E_{\mathrm{id}}^{(\mathsf{P2})}(\mathit{len}_x^{\mathrm{B64}}),\;
	E_{\mathrm{id}}^{(\mathsf{P2})}(\mathit{drop}_x),\;
	E_{\mathrm{id}}^{(\mathsf{P2})}(\mathit{len}_x^{\mathrm{inner}}), \\[2pt]
	& E_{\mathrm{id}}^{(\mathsf{P2})}(\mathit{off}_y^{\mathrm{B64}}),\;
	E_{\mathrm{id}}^{(\mathsf{P2})}(\mathit{len}_y^{\mathrm{B64}}),\;
	E_{\mathrm{id}}^{(\mathsf{P2})}(\mathit{drop}_y),\;
	E_{\mathrm{id}}^{(\mathsf{P2})}(\mathit{len}_y^{\mathrm{inner}})
	\,\} \subseteq \mathcal{E}_{\mathsf{P2}}.
\end{aligned}
\]

The recursive proof artifacts $\mathcal{A}_{\mathrm{inner}}=(\pi_{\text{cred-bind}},\mathsf{vk}_{\text{cred-bind}},\mathsf{pp}_{\text{cred-bind}})$ are produced and consumed internally by the Plonky2 toolchain and therefore remain outside the scope of $\mathsf{PreProc}$. 
\section{Monolithic implementation (Groth16)}
\label{sec:monolithic}
We begin by realizing the credential–wallet binding proof $\pi_{\text{cred-bind}}$ from Chapter~\ref{chap:construction} as a single Groth16 circuit, compiled with the \texttt{Circom}\footnote{\url{https://github.com/iden3/circom}, version~2.2.2} compiler and executed through the \texttt{snarkjs}\footnote{\url{https://github.com/iden3/snarkjs}, version~0.7.5} and \texttt{RapidSnark}\footnote{\url{https://github.com/iden3/rapidsnark}, version~0.0.7} proving backends. This monolithic instantiation serves as the baseline of our evaluation. By monolithic we mean that all four constraints of $\pi_{\text{cred-bind}}$, namely correctness of the credential key ($C_{\text{cred-bind}}^{(1)}$), credential ownership ($C_{\text{cred-bind}}^{(2)}$), credential validity ($C_{\text{cred-bind}}^{(3)}$), and blockchain wallet control ($C_{\text{cred-bind}}^{(4)}$), are enforced jointly in a single circuit $C_{\text{cred-bind}}$, which outputs a succinct Groth16 proof. The resulting proof demonstrates in a single argument that the prover possesses a valid issuer-signed credential and controls a blockchain wallet key. This section outlines how the constraints are arithmetized into circuit components and highlights optimizations applied to reduce their cost.

\paragraph{Groth16 prover}
As outlined in Section~\ref{sec:zksnarks}, zk-SNARK statements are expressed as arithmetic circuits that can be compiled into structured constraint systems. Groth16 in particular represents such circuits as a \acrfull{qap}, which enables succinct non-interactive arguments of knowledge. Concretely, the circuit $C_{\text{cred-bind}}$ is compiled into a polynomial system of degree proportional to the number of constraints $N$, which is then arithmetized into a QAP~\cite{groth2016size}. The prover’s workload is dominated by \acrfull{msm} operations on elliptic-curve groups and \acrfull{fft} operations over field polynomials of degree $N$~\cite{groth2016size,cryptoeprint:2019/1047}. Asymptotically, the prover complexity can be written as:
\[
T_{\text{prover}}^{\text{Groth16}}(N) 
\approx \alpha_{\text{MSM}} \cdot N + \beta_{\text{FFT}} \cdot N\log N.
\]
Here, the first term stems from \acrshort{msm}s on the proving key and the second from \acrshort{fft}-based polynomial evaluation. In contrast to recursive proof systems, Groth16 proofs have \emph{constant} size (three group elements) and admit a verifier runtime independent of $N$, consisting only of $O(|x|)$ exponentiations, where $|x|$ denotes the length of the public input, plus three elliptic-curve pairings~\cite{groth2016size}. Since pairings are generally significantly more expensive than exponentiations, the verifier’s cost is effectively dominated by these three pairings~\cite{boneh2020moderncrypto}. These properties make Groth16 particularly attractive for monolithic circuits with moderate constraint counts and explain why the Circom--snarkjs--RapidSnark toolchain is a competitive baseline for evaluating our construction.

\paragraph{Arithmetization and components}
The circuit $C_{\text{cred-bind}}$ enforces the four constraints of the construction by mapping them into polynomial equalities over the base field $\mathbb{F}$ of the proof system. Each constraint corresponds to a relation $R_i \subseteq \mathcal{X}_i \times \mathcal{W}_i$, and their conjunction defines the relation:
\[
R_{\text{cred-bind}} = \{(x,w) : \bigwedge_{i=1}^4 f_{C_{\text{cred-bind}}^{(i)}}(x,w)=0\}.
\]
In this equation, $f_{C_{\text{cred-bind}}^{(i)}}(x,w)=0$ denotes a vector of polynomial constraints associated with the $i$-th constraint, and every polynomial in this vector must evaluate to zero. All elliptic-curve operations are defined over the prime fields $\mathbb{F}_p$ (for secp256r1) and $\mathbb{F}_{p'}$ (for secp256k1), but since the Groth16 backend operates over the SNARK field $\mathbb{F}$ (BN254), these operations are realized in-circuit via limb-decomposed encodings. That is, integers modulo $p$ or $p'$ are represented by vectors of field elements in $\mathbb{F}$, and modular reduction is enforced through explicit constraints.

\medskip
Constraint~$C_{\text{cred-bind}}^{(1)}$ ensures correctness of the credential key on secp256r1 by enforcing:
\[
pk_c = [sk_c]G \in E(\mathbb{F}_p).
\]
Here, $G$ is the generator of secp256r1 over $\mathbb{F}_{p}$ with $p = 2^{256}-2^{224}+2^{192}+2^{96}-1$
~\cite{SEC2}. The \emph{scalar multiplication} is decomposed into a sequence of point additions and doublings, where for each bit $b_j$ of $sk_c$ one enforces:
\[
(x_{j+1},y_{j+1})=(x_j,y_j)+b_j\cdot (2^j G).
\]
The group law is arithmetized by the Weierstraß equations:
\[
\lambda = (y_2-y_1)\cdot(x_2-x_1)^{-1}, \quad x_3=\lambda^2-x_1-x_2, \quad y_3=\lambda(x_1-x_3)-y_1.
\]

\medskip
Constraint~$C_{\text{cred-bind}}^{(2)}$ guarantees consistency between the derived point from $C_{\text{cred-bind}}^{(1)}$ and the public key embedded in the \acrshort{eudi} credential payload. To this end, the circuit reconstructs the public key coordinates $(x_c^{(\mathcal{C})},y_c^{(\mathcal{C})})$ directly from the base64url-encoded payload. Formally, let $\mathrm{payloadB64}=(s_0,\dots,s_{L_p-1}) \in \Sigma_{\mathrm{b64}}^{L_p}$ with $\Sigma_{\mathrm{b64}}=\{A\!-\!Z,a\!-\!z,0\!-\!9,-,\_\,,=\}$. For each coordinate $u \in \{x,y\}$, the preprocessing provides slicing parameters $(\mathit{off}_u^{\mathrm{B64}}, \mathit{len}_u^{\mathrm{B64}}, \mathit{drop}_u, \mathit{len}_u^{\mathrm{inner}})$ that specify how the relevant string is to be decoded. The circuit first selects a 64-character slice $S_u=(s_{\mathit{off}_u^{\mathrm{B64}}},\dots,s_{\mathit{off}_u^{\mathrm{B64}}+\mathit{len}_u^{\mathrm{B64}}-1})$ with $\mathit{len}_u^{\mathrm{B64}}=64$ and checks that all symbols lie in $\Sigma_{\mathrm{b64}}$. The choice of $64$ characters is deliberate. Since each base64url character represents 6 bits, groups of four characters yield exactly $4\cdot 6=24$ bits, corresponding to three bytes. A 64-character slice therefore consists of $64/4=16$ blocks, which decode to $16\cdot 3=48$ bytes. This ensures alignment with the block structure of the encoding and guarantees that the entire coordinate substring is contained within the slice. From the resulting 48-byte sequence, the first $\mathit{drop}_u$ bytes are discarded, because the decoded output may still contain irrelevant JSON key material such as \texttt{"x"} or \texttt{"y"}. The remaining part begins exactly at the start of the base64url-encoded coordinate. From there, $\mathit{len}_u^{\mathrm{inner}}\in\{43,44\}$ characters are retained. If $\mathit{len}_u^{\mathrm{inner}}=43$, the padding character `=' is appended to obtain a canonical 44-character representation. This is necessary since a 256-bit coordinate (32 bytes) encodes to either 43 characters plus a padding symbol, or 44 characters without padding. In both cases the canonical form has length 44. The 44-character inner string is decoded again using the same $4\mapsto 3$ rule, where every four base64url characters yield three bytes. With 44 characters this results in $44/4=11$ blocks, producing $11\cdot 3=33$ bytes in total. Depending on whether the last symbol of the inner string was `=', the circuit selects either the first 32 bytes or the final 32 bytes of this sequence. The selected 32-byte window is then interpreted as a big-endian integer:
\[
u_c^{(\mathcal{C})} = \sum_{j=0}^{31} b_j \cdot 256^{31-j} \in [0,2^{256}).
\]
In this equation, $b_0$ denotes the most significant byte. This yields the affine coordinate value embedded in the credential. Finally, the pair of decoded coordinates is compared against the point derived in $C_{\text{cred-bind}}^{(1)}$ by enforcing:
\[
f_{C_{\text{cred-bind}}^{(2)}}(X) =
\bigl(x_c^{(C_{\text{cred-bind}}^{(1)})} - x_c^{(\mathcal{C})},\;
y_c^{(C_{\text{cred-bind}}^{(1)})} - y_c^{(\mathcal{C})}\bigr) = (0,0).
\]

Constraint~$C_{\text{cred-bind}}^{(3)}$ realizes \acrshort{ecdsa} verification on secp256r1 with respect to the issuer’s public key $Q_I \in E(\mathbb{F}_p)$. The circuit first computes the message hash internally from the credential contents. Let \textit{headerB64} and \textit{payloadB64} denote the base64url-encoded header and payload strings of the credential, and let $\Sigma_{\mathrm{ascii}}$ denote the ASCII alphabet. The circuit concatenates these strings with the delimiter symbol `.', yielding the byte sequence:
\[
M = \texttt{ASCII}(\textit{headerB64}) \,\|\, \texttt{ASCII}(\texttt{'.'}) \,\|\, \texttt{ASCII}(\textit{payloadB64})
\;\in\; \Sigma_{\mathrm{ascii}}^{L_h + 1 + L_p}.
\]
Here, $L_h$ and $L_p$ denote the respective lengths. It then computes the
SHA-256 digest of $M$, represented as an integer:
\[
h = \mathrm{SHA256}(M) \in \mathbb{Z}_n.
\]
More precisely, $n$ denotes the group order of secp256r1. The SHA-256 computation is realized in-circuit through the standard compression function arithmetization using bit-decomposed state words. Given this value $h$, a signature $(r,s)\in\mathbb{Z}_n^2$, and the issuer’s public key $Q_I$, verification requires:
\[
R = [u_1]G + [u_2]Q_I, \qquad r \equiv x(R) \pmod{n}.
\]
Specifically, $u_1=h\cdot s^{-1}\bmod n$ and $u_2=r\cdot s^{-1}\bmod n$. This follows the algorithm as specified in FIPS~186-5~\cite{FIPS186-5}. The corresponding arithmetization enforces the system:
\[
\begin{aligned}
	& s\cdot s^{-1} - 1 \equiv 0 \pmod{n} \;\wedge \\
	& u_1 - h\cdot s^{-1} \equiv 0 \pmod{n} \;\wedge \quad u_2 - r\cdot s^{-1} \equiv 0 \pmod{n} \;\wedge \\
	& (x_{u_1},y_{u_1}) = [u_1]G \;\wedge \quad (x_{u_2},y_{u_2}) = [u_2]Q_I \;\wedge \\
	& (x_R,y_R) = (x_{u_1},y_{u_1}) + (x_{u_2},y_{u_2}) \;\wedge \\
	& x_R - r \equiv 0 \pmod{n}.
\end{aligned}
\]
As in $C_{\text{cred-bind}}^{(1)}$, the scalar multiplications and point addition are decomposed into polynomial constraints via the Weierstraß addition law. In this way, the circuit ensures that the signature $(r,s)$ is valid for the message digest derived from the actual credential contents. Constraint~$C_{\text{cred-bind}}^{(4)}$ ensures correctness of the blockchain wallet key on secp256k1 by enforcing:
\[
pk_0 = [sk_0]G \in E(\mathbb{F}_{p'}).
\]
Here, $G$ is the generator of secp256k1 over $\mathbb{F}_{p'}$ with $p' = 2^{256}-2^{32}-2^9-2^8 -2^7-2^6-2^4-1$~\cite{SEC2}. The scalar multiplication is expressed analogously to $C_{\text{cred-bind}}^{(1)}$ using addition and doubling formulas, but instantiated over $\mathbb{F}_{p'}$.

\medskip
Taken together, these four components yield a system of polynomial constraints that define the relation $R_{\text{cred-bind}}$ and constitute the input to the Groth16 compilation step, which transforms the system into an \acrshort{r1cs} instance and further into a \acrshort{qap} of degree proportional to the number of constraints.

\paragraph{Fixed-base scalar multiplication optimization}
A central cost driver in constraint~$C_{\text{cred-bind}}^{(3)}$ is the computation of $[u_2]Q_I$, where $Q_I$ denotes the issuer’s public key. In the generic setting, $Q_I$ is treated as a variable circuit input and thus requires a \emph{variable-base} scalar multiplication. Formally, this operation is realized by decomposing the scalar $u_2 \in \mathbb{Z}_n$ into its bit representation $u_2 = \sum_{j=0}^{\lambda-1} b_j 2^j$ with $b_j \in \{0,1\}$, and enforcing iteratively:
\[
R_{j+1} = 2R_j + b_j Q_I, \qquad R_0 = \mathcal{O}.
\]
In this equation, $\mathcal{O}$ is the point at infinity and $\lambda=256$ is the scalar bit length. This procedure induces $\lambda$ curve doublings and, on average, $\frac{\lambda}{2}$ conditional additions, each expressed through non-native modular arithmetic over $\mathbb{F}_{p}$. Consequently, the variable-base multiplication dominates the overall circuit size, contributing about $65\%$ of all non-linear constraints in the unoptimized variant.

Our construction admits a stronger assumption. The issuer’s public key $Q_I$ is unique and fixed across all proofs. This follows from the trust model of a single credential issuer introduced in Chapter~\ref{chap:construction}. We therefore embed $Q_I$ directly into the circuit as a constant. This enables the use of \emph{fixed-base} scalar multiplication, in which all multiples of $Q_I$ are precomputed and stored in a lookup table. The scalar multiplication $[u_2]Q_I$ can then be expressed as a sequence of table lookups and a logarithmic number of additions, instead of a full bitwise double-and-add traversal\footnote{The implementation of this optimization is available in our fork of the \texttt{circom-ecdsa-p256} library: \url{https://github.com/sommer-ph/circom-ecdsa-p256}.}. The underlying windowing technique is a well-established optimization for elliptic-curve scalar multiplication, which we apply here to reduce the cost of $[u_2]Q_I$~\cite{boneh2020moderncrypto}. Concretely, let $u_2 = \sum_{i=0}^{t-1} v_i 2^{si}$ be the scalar decomposition into $t=\lceil \frac{\lambda}{s}\rceil$ windows of size $s$ bits. For each window position $i$ and each value $v \in \{0,\dots,2^s-1\}$ we precompute
\[
P_{i,v} = v \cdot 2^{si} Q_I \in E(\mathbb{F}_p).
\]
The fixed-base multiplication proceeds in three steps, which are formally detailed in Appendix~\ref{app:fixed-base}. First, the scalar $u_2$ is decomposed into $t$ windows $(v_0,\dots,v_{t-1})$ of $s$ bits each. Second, for every index $i$, the corresponding precomputed point $M_i = P_{i,v_i}$ is selected using multiplexer constraints. Finally, the resulting points are aggregated to obtain:
\[
R = \sum_{i=0}^{t-1} M_i.
\]
This computation requires $t-1$ elliptic-curve additions. The correctness follows immediately since the precomputation covers all possible window values:
\[
\sum_{i=0}^{t-1} P_{i,v_i}
= \sum_{i=0}^{t-1} v_i \cdot 2^{si} Q_I
= \left(\sum_{i=0}^{t-1} v_i 2^{si}\right) Q_I
= [u_2]Q_I.
\]

\medskip
The cost reduction can be captured formally by the following model.
Let $\alpha_{\mathrm{add}},\alpha_{\mathrm{dbl}},\alpha_{\mathrm{mux}}$ denote the constraint costs of a point addition, doubling, and multiplexer step, respectively. The unoptimized (variable-base) multiplication incurs:
\[
\mathrm{cost}_{\mathrm{var}} \approx \alpha_{\mathrm{dbl}}\cdot N_{\mathrm{dbl}} + \alpha_{\mathrm{add}}\cdot N_{\mathrm{add}},
\qquad
N_{\mathrm{dbl}} \approx \lambda, \;\; N_{\mathrm{add}} \approx \tfrac{\lambda}{2}.
\]
In this expression, every bit requires one doubling and, on average, one addition every second bit. In contrast, the fixed-base multiplication incurs:
\[
\mathrm{cost}_{\mathrm{fix}} \approx \alpha_{\mathrm{add}}\cdot (t-1) + \alpha_{\mathrm{mux}}\cdot t \cdot 2^s.
\]
Here, $t=\lceil \frac{\lambda}{s}\rceil$ is the number of windows. For $s=8$ and $\lambda=256$ we obtain $t=32$, eliminating all doublings and reducing the number of additions to $31$. The overhead stems only from the multiplexer logic needed to select the correct precomputed point.

\medskip
Table~\ref{tab:fixed-base-comparison} summarizes the empirical effect of this optimization. We observe a reduction from $2.54 \cdot 10^6$ non-linear constraints in the baseline to about $8.10 \cdot 10^5$ in the optimized circuit. This corresponds to a proof generation speedup by a factor of roughly $3.5$, while the verification time remains essentially unaffected. Inspecting the optimized circuit in more detail reveals that the majority of the remaining cost stems from the in-circuit SHA-256 computation. In isolation, the SHA-256 gadget accounts for about $5.66 \cdot 10^5$ non-linear and $2.26 \cdot 10^4$ linear constraints, i.e.\ more than two thirds of the total size of $C_{\text{cred-bind}}^{(3)}$.

\begin{table}[t]
	\centering
	\begin{tabular}{lrrr}
		\toprule
		& \textbf{Baseline (variable-base)} & \textbf{Optimized (fixed-base)} & \textbf{Improvement} \\
		\midrule
		Non-linear constraints & 2\,539\,133 & 810\,138 & $\times 3.13$ \\
		Linear constraints     & 255\,024   & 240\,513 & $\times 1.06$ \\
		\midrule
		Proof generation       & 32.52\,s  & 9.35\,s & $\times 3.48$ \\
		Proof verification     & 0.02\,s   & 0.02\,s & --- \\
		\bottomrule
	\end{tabular}
	\caption{Impact of fixed-base optimization on the ECDSA verification (constraint~$C_{\text{cred-bind}}^{(3)}$).}
	\label{tab:fixed-base-comparison}
\end{table}

\medskip
\noindent\emph{Security considerations.} 
The precomputation table is derived exclusively from the public constant $Q_I$, therefore no secret information is embedded in the circuit. Correctness is guaranteed by the completeness of the lookup table, which covers all possible window values $v \in \{0,\dots,2^s-1\}$, and by the fact that all multiplexer constraints are realized in constant form, preserving side-channel resistance. Moreover, the circuit enforces range and invertibility conditions on the signature scalars $(r,s)$. Specifically, $0 < r,s < n$ via range checks and $s \neq 0$ via modular inversion. Elliptic-curve consistency is ensured since the invoked group-operation templates (\texttt{Double}, \texttt{AddUnequal}, \texttt{ScalarMult}) internally enforce the Weierstraß addition and doubling laws modulo $p$, thereby guaranteeing that all computed points lie on $E(\mathbb{F}_p)$ and satisfy the group law. As $Q_I$ is embedded as a compile-time constant on the curve secp256r1 with cofactor $h=1$, no additional subgroup validation is required~\cite{SEC2}. The only functional trade-off is the rigidity of binding to a single issuer key, which is consistent with our system model but would need to be reconsidered in multi-issuer deployments.

\paragraph{Performance analysis}
Table~\ref{tab:fixed-base-comparison} quantified the local effect of the fixed-base optimization on constraint~$C_{\text{cred-bind}}^{(3)}$. We now extend this evaluation to the entire monolithic construction $\pi_{\text{cred-bind}}$ and to its individual subcomponents. In addition to the structural parameters (linear and non-linear constraints), we report pipeline metrics comprising circuit compilation, witness generation, trusted setup, verification key export, proof generation, proof verification, and proof size. Measurements are given for the unoptimized baseline, the optimized variant with fixed-base scalar multiplication, and, for completeness, the experimental PLONK backend of \texttt{snarkjs}. To attribute costs more precisely, we also evaluate the isolated constraints ($C_{\text{cred-bind}}^{(1)}$-$C_{\text{cred-bind}}^{(4)}$) and contrast these with SNARK-friendly benchmarks over BabyJubJub. This twisted Edwards curve is defined over the scalar field of BN254, which avoids non-native field operations and serving as a SNARK-friendly benchmark. A formal derivation and security analysis of BabyJubJub is given in \cite{BellesMunoz2021}. This multi-level analysis provides both absolute performance data and a decomposition of the dominant cost factors in $\pi_{\text{cred-bind}}$.

The results of Table~\ref{tab:monolithic} highlight the quantitative effect of the fixed-base optimization when applied to the monolithic construction $\pi_{\text{cred-bind}}$. All performance tables are obtained with RapidSnark for proof generation and verification, and all times are reported in seconds and proof sizes in bytes. The unoptimized variant incurs about $3.05 \cdot 10^6$ non-linear constraints, leading to a median proof generation time of $34.12\,s$. By contrast, the optimized variant reduces the constraint count by more than half ($1.32 \cdot 10^6$ non-linear constraints) and achieves a proof generation time of $8.24\,s$, corresponding to a speedup factor of about four. Circuit compilation, witness generation, and trusted setup still dominate the one-time preprocessing costs. In our measurements, circuit compilation required $99.96$--$133.93$\,s, witness generation $65.93$--$282.01$\,s, and the trusted setup $504.42$--$1059.73$\,s, depending on the backend and optimization. These costs are amortized once the proving key is fixed. For repeated executions in realistic deployments, only proof generation and verification are relevant, and these stabilize at the aforementioned $8.24\,s$ and $0.01\,s$, respectively. 
As described previously, both proof size and verification time in Groth16 are independent of the circuit size. Our measurements confirm this behavior. Proofs remain constant at about $805$~bytes with minor variations due to RapidSnark serialization, and verification time is stable around $0.01$\,s. Consequently, the evaluation focuses on prover-side costs, which dominate the overall performance.

For completeness, we also evaluated the PLONK backend of \texttt{snarkjs}. Here proof generation required $1367.62\,s$, two orders of magnitude slower than Groth16. We attribute this gap not to an inherent inefficiency of PLONK, but to the current implementation maturity of \texttt{snarkjs}. Other PLONK-style systems such as Plonky2 can also deliver competitive performance, as will be shown in Section~\ref{sec:recursive-plonky2}. The measurements in Table~\ref{tab:monolithic} should therefore be interpreted as representative for Groth16 with RapidSnark, while the PLONK figures serve only as an implementation baseline.

\begin{table}[t]
	\centering
	\begin{tabular}{lrrr}
		\toprule
		& \textbf{Monolithic (unopt.)} & \textbf{Monolithic (opt.)} & \textbf{Monolithic (opt., PLONK)} \\
		\midrule
		Non-linear constraints & 3\,049\,865 & 1\,320\,870 & 454\,105 \\
		Linear constraints     &   443\,879 &   429\,368 & 396\,052 \\
		\midrule
		Circuit compilation    & 133.93\,s  & 99.96\,s   & 40.92\,s \\
		Witness generation     & 282.01\,s  & 65.93\,s   & 58.90\,s \\
		Trusted setup          & 1059.73\,s & 504.42\,s  & 1445.32\,s \\
		Key export             & 0.87\,s    & 0.69\,s    & 1.02\,s \\
		Proof generation       & 34.12\,s   & 8.24\,s    & 1367.62\,s \\
		Proof verification     & 0.02\,s    & 0.01\,s    & 0.64\,s \\
		Proof size             & 805\,B     & 803\,B     & 805\,B \\
		\bottomrule
	\end{tabular}
	\caption{Metrics of the monolithic implementation $\pi_{\text{cred-bind}}$.}
	\label{tab:monolithic}
\end{table}

The decomposition in Table~\ref{tab:monolithic-subcircuits} attributes the cost of $\pi_{\text{cred-bind}}$ to its individual constraints. All measurements were obtained with RapidSnark for proof generation and verification, and all times are reported in seconds and proof sizes in bytes. Constraint~$C_{\text{cred-bind}}^{(2)}$ enforces consistency of the credential public key by comparing the value embedded in the \acrshort{eudi} credential payload with the one derived in~$C_{\text{cred-bind}}^{(1)}$. The actual comparison is trivial, but the circuit must internally decode the base64url-encoded coordinates from the payload. As a result, almost the entire cost of $C_{\text{cred-bind}}^{(2)}$ stems from this decoding step, which accounts for roughly $3.0 \cdot 10^5$ non-linear constraints and a $2.12\,s$ proof time. Constraint~$C_{\text{cred-bind}}^{(3)}$ combines in-circuit hashing and signature verification. Specifically, the circuit computes the SHA-256 digest of the credential header and payload and then verifies the attached \acrshort{ecdsa} signature against the issuer’s public key $Q_I$. This subcircuit is the dominant contributor, with $2.54 \cdot 10^6$ non-linear constraints in the unoptimized variant and a proof generation time of $32.52\,s$. Applying the fixed-base optimization to the multiplication by $Q_I$ reduces the cost to $8.1 \cdot 10^5$ constraints and $9.35\,s$ proof time, thereby accounting for the global performance gain. The key derivation constraints $C_{\text{cred-bind}}^{(1)}$ and $C_{\text{cred-bind}}^{(4)}$ are substantially lighter, each contributing around $10^5$ constraints and $1.10$--$2.27\,s$ proving time. Structurally, $C_{\text{cred-bind}}^{(1)}$ and $C_{\text{cred-bind}}^{(4)}$ each implement a single fixed-base scalar multiplication with the generator $G$, whereas $C_{\text{cred-bind}}^{(3)}$ combines two such multiplications (for $G$ and $Q_I$) with an elliptic-curve addition, modular inversion, additional order-field arithmetic, and the in-circuit computation of the SHA-256 hash. This explains why $C_{\text{cred-bind}}^{(3)}$ remains by far the most expensive component, exceeding $C_{\text{cred-bind}}^{(1)}$ or $C_{\text{cred-bind}}^{(4)}$ even after optimization.
It should be noted that while the constraint counts of the subcircuits add up consistently to the global circuit size, the pipeline timings are not strictly additive. The Groth16 prover applies shared preprocessing such as \acrshort{fft} domain sizing and multi-scalar multiplication batching, which introduces nonlinear interactions between subcircuits. Subcircuit measurements should therefore be read as relative indicators of cost contribution, not as components of an additive runtime model. The key insight remains that signature verification, together with the in-circuit hashing, dominates both structurally and in prover time, and that fixed-base scalar multiplication is the decisive primitive across most subcircuits.

\begin{table}[t]
	\centering
	\begin{tabular}{lrrrrr}
		\toprule
		& \textbf{$C_{\text{cred-bind}}^{(1)}$} 
		& \textbf{$C_{\text{cred-bind}}^{(2)}$} 
		& \textbf{$C_{\text{cred-bind}}^{(3)}$} 
		& \textbf{$C_{\text{cred-bind}}^{(3)}$ (opt.)} 
		& \textbf{$C_{\text{cred-bind}}^{(4)}$} \\
		\midrule
		Non-linear constraints & 114\,724 & 300\,564 & 2\,539\,133 & 810\,138 & 95\,444 \\
		Linear constraints     & 108\,599 & 10\,766  & 255\,024   & 240\,513 & 69\,490 \\
		\midrule
		Circuit compilation    & 12.02\,s & 8.32\,s  & 86.68\,s   & 59.70\,s & 9.97\,s \\
		Witness generation     & 12.69\,s & 0.62\,s  & 245.88\,s  & 33.58\,s & 26.04\,s \\
		Trusted setup          & 32.02\,s & 38.12\,s & 1016.27\,s & 334.19\,s & 41.04\,s \\
		Key export             & 0.40\,s  & 0.44\,s  & 0.66\,s    & 0.47\,s  & 0.47\,s \\
		Proof generation       & 1.10\,s  & 2.12\,s  & 32.52\,s   & 9.35\,s  & 2.27\,s \\
		Proof verification     & 0.01\,s  & 0.01\,s  & 0.02\,s    & 0.02\,s  & 0.01\,s \\
		Proof size             & 806\,B   & 806\,B   & 805\,B     & 806\,B   & 806\,B \\
		\bottomrule
	\end{tabular}
	\caption{Subcircuit metrics of the monolithic implementation $\pi_{\text{cred-bind}}$.}
	\label{tab:monolithic-subcircuits}
\end{table}

Table~\ref{tab:monolithic-native} reports three sanity baselines over BabyJubJub. These benchmarks are not intended as direct competitors to our target construction, but rather as reference points to highlight the efficiency of native arithmetic in the SNARK field and to isolate protocol-level overhead. They should not be read as a cryptographic comparison between \acrshort{ecdsa} and \acrfull{eddsa}. A fair comparison would require implementing both schemes over the same curve. Concretely, we evaluate a fixed-base key derivation, an \acrshort{eddsa} verification without hashing, and an \acrshort{eddsa} verification where the message is obtained by 
computing the in-circuit SHA-256 digest over the base64url encoded \acrshort{eudi} credential header and payload, as in $C_{\text{cred-bind}}^{(3)}$. All circuits execute entirely over the scalar field of BN254. The absence of non-native field emulation explains the small constraint profile of key derivation, which requires only $3{,}939$ non-linear constraints and $183$ linear constraints, leading to $0.10$\,s for proof generation. \acrshort{eddsa} verification alone adds one variable-base and one fixed-base scalar multiplication on BabyJubJub, requiring $8{,}873$ non-linear constraints and $200$ linear constraints, resulting in $0.16$\,s for proof generation. By contrast, when SHA-256 hashing of the credential input is included, the cost increases by nearly two orders of magnitude to $575{,}080$ non-linear and $22{,}762$ linear constraints, leading to $4.51$\,s for proof generation. Comparing these baselines with the non-native subcircuits in Table~\ref{tab:monolithic-subcircuits} makes two effects visible, which are described below.

\medskip
The first effect is that non-native field emulation dominates. To quantify the overhead introduced by non-native arithmetic for the key derivation primitive, we compare the number of non-linear constraints of the subcircuits (Tables~\ref{tab:monolithic-subcircuits} and~\ref{tab:monolithic-native}). Using the constraint counts as cost indicators, we obtain:
\[
\frac{\text{$C_{\text{cred-bind}}^{(1)}$ (secp256r1)}}{\text{KeyDer (BabyJubJub)}} \approx 29.1,\qquad
\frac{\text{$C_{\text{cred-bind}}^{(4)}$ (secp256k1)}}{\text{KeyDer (BabyJubJub)}} \approx 24.2.
\]
For signature verification the effect is smaller once hashing is aligned. Comparing the optimized $C_{\text{cred-bind}}^{(3)}$ with \acrshort{eddsa} over BabyJubJub including in-circuit SHA-256 yields:
\[
\frac{\text{$C_{\text{cred-bind}}^{(3)}$ (opt., secp256r1 \acrshort{ecdsa})}}{\text{\acrshort{eddsa} (BabyJubJub, with SHA-256)}} \approx 1.4.
\]

These factors arise because every field operation over the external curve modulus ($p_{\text{secp256r1}}$ or $p_{\text{secp256k1}}$) must be realized in BN254 via limb decomposition and constrained modular reduction. In the libraries used here this corresponds to $6\times 43$-bit limbs for secp256r1 and $4\times 64$-bit limbs for secp256k1. The choice of limb size reflects library design trade-offs. Smaller limbs (43\,bits) reduce range-check complexity but require more limbs, whereas larger limbs (64\,bits) increase the per-limb range checks but reduce the number of multiplication chains. In both cases, each multiplication or division expands into multiple multi-limb products, carry constraints, and range checks, inflating the non-linear and linear constraint counts. Second, protocol overhead in \acrshort{ecdsa} remains after fixed-base optimizations. Even with both scalar multiplications implemented as fixed-base windows (for $G$ and the issuer public key $Q_I$), \acrshort{ecdsa} verification includes extra order-arithmetic that \acrshort{eddsa} on BabyJubJub avoids. In particular, computing $s^{-1}\bmod n$, forming $u_1 = h\cdot s^{-1}\bmod n$ and $u_2 = r\cdot s^{-1}\bmod n$, and performing an additional elliptic-curve addition to combine $[u_1]G$ and $[u_2]Q_I$. In our non-native stack, modular inversion is realized by exponentiation (Fermat) over the group order and thus translates to a long chain of modular products and reductions. Moreover, the Weierstraß group law (\texttt{PointAdd}, \texttt{PointDouble}) itself requires computing slope parameters involving modular division, which again expands into non-native inverse constraints. While both \acrshort{ecdsa} and our \acrshort{eddsa} baseline include in-circuit SHA-256 hashing over the credential input, the additional order-arithmetic and non-native field operations in \acrshort{ecdsa} lead to a higher constraint count. Consequently, the optimized \acrshort{ecdsa} subcircuit $C_{\text{cred-bind}}^{(3)}$ still exceeds the cost of \acrshort{eddsa} verification, even though both share hashing and elliptic-curve scalar multiplications as building blocks.

\medskip
The constraint gaps above mirror proving time. BabyJubJub key derivation at $0.10$\,s contrasts with $1.10$\,s ($C_{\text{cred-bind}}^{(1)}$, secp256r1) and $2.27$\,s ($C_{\text{cred-bind}}^{(4)}$, secp256k1). For signature verification, BabyJubJub \acrshort{eddsa} with SHA-256 at $4.51$\,s contrasts with $9.35$\,s ($C_{\text{cred-bind}}^{(3)}$, optimized secp256r1). While these timings are not strictly additive across subcircuits, they reliably indicate where the prover spends work. Non-native arithmetic inflates every field operation by a limb-level factor, and \acrshort{ecdsa}’s order-field inversion, modular multiplications, and additional elliptic-curve additions contribute a protocol-specific overhead on top of the scalar multiplications. Together, these two effects account for the large constraint and proving-time gaps between the non-native (secp256r1, secp256k1) and native (BabyJubJub) instantiations.


\begin{table}[t]
	\centering
	\begin{tabular}{lrrr}
		\toprule
		& \textbf{Key derive} 
		& \textbf{\acrshort{eddsa} verify} 
		& \textbf{\acrshort{eddsa} verify with SHA-256} \\
		\midrule
		Non-linear constraints & 3\,939   & 8\,873    & 575\,080 \\
		Linear constraints     & 183      & 200      & 22\,762 \\
		\midrule
		Circuit compilation    & 0.20\,s  & 0.71\,s  & 66.68\,s \\
		Witness generation     & 0.08\,s  & 0.16\,s  & 3.94\,s \\
		Trusted setup          & 3.82\,s  & 8.68\,s  & 283.54\,s \\
		Key export             & 0.37\,s  & 0.50\,s  & 0.60\,s \\
		Proof generation       & 0.10\,s  & 0.16\,s  & 4.51\,s \\
		Proof verification     & 0.01\,s  & 0.01\,s  & 0.02\,s \\
		Proof size             & 806\,B   & 805\,B   & 804\,B \\
		\bottomrule
	\end{tabular}
	\caption{SNARK-friendly comparison metrics over BabyJubJub.}
	\label{tab:monolithic-native}
\end{table}

\section{Recursive implementation (Plonky2)}
\label{sec:recursive-plonky2}
We realize the derived key binding proof $\pi_{\text{key-bind}}$ from Chapter~\ref{chap:construction} as a two-layer recursive SNARK in \texttt{Plonky2}\footnote{\url{https://github.com/0xPolygonZero/plonky2}}. The \emph{inner circuit} $C_{\text{cred-bind}}$ enforces the same credential–wallet binding constraints as the monolithic variant and outputs a proof $\pi_{\text{cred-bind}}$ with public inputs $(\mathit{pk}_I,\mathit{pk}_0)$. The \emph{outer circuit} $C_{\text{key-bind}}$ embeds the Plonky2 verifier gadget to check $\pi_{\text{cred-bind}}$ and, in addition, enforces the non-hardened derivation relation for a child public key, yielding a recursive proof with public inputs $(\mathit{pk}_I,\mathit{pk}_i,\mathit{cc}_0,i)$. The resulting proof demonstrates, in a single argument, that the prover controls a child public key $\mathit{pk}_i$ that is correctly derived from a parent key $\mathit{pk}_0$, and that this parent key has been bound to a valid issuer-signed \acrshort{eudi} credential via $\pi_{\text{cred-bind}}$, without revealing any secret information. This section outlines how the inner and outer circuits are arithmetized in Plonky2 into circuit components, describes optimizations applied to reduce their cost, and presents a performance evaluation.

\paragraph{Plonky2 prover}
Plonky2 follows a different design paradigm than Groth16. It does not require a \acrfull{srs} and instead realizes a transparent proof system in the PLONK family, combined with a \acrfull{fri}-based polynomial commitment scheme~\cite{cryptoeprint:2019/953,cryptoeprint:2018/046}. Statements are expressed as arithmetic circuits over the Goldilocks field, which are compiled into a polynomial \acrfull{iop}. The prover commits to evaluations of low-degree extensions of the witness polynomials using Merkle trees, and correctness is argued through a sequence of \acrshort{fri} queries that reduce the degree of the polynomial step by step~\cite{cryptoeprint:2019/1076}. Unlike Groth16, where the workload is dominated by \acrfull{msm} operations on elliptic curves, the Plonky2 prover is dominated by \acrshort{fft}s over the evaluation domain and Merkle-tree commitments for the polynomial oracles. In practice, hashing and Merkle-tree operations form the main performance bottleneck, and for recursive proofs, verifying Merkle paths alone accounts for the majority of the verifier circuit~\cite{Plonky2Draft2022}. Asymptotically, the prover complexity can be \emph{heuristically} approximated as:
\[
T_{\text{prover}}^{\text{Plonky2}}(N) 
\approx \alpha_{\text{FFT}} \cdot N\log N 
+ \beta_{\text{Merkle}} \cdot N\log N 
+ \gamma_{\text{FRI}} \cdot N.
\]
In this equation, $N$ denotes the circuit size in gates, the first term reflects the \acrshort{fft} cost for polynomial evaluation and low-degree extension, the second term corresponds to Merkle-commitment construction, and the third term collects the linear costs of \acrshort{fri} reductions (query sampling and low-degree tests)~\cite{cryptoeprint:2018/046}. A key property of Plonky2 is that both its prover and verifier are designed to be \emph{recursion-friendly}. Proofs can be efficiently verified inside another circuit, since the outer verifier only needs to re-check the \acrshort{fri} queries and Merkle openings of the inner proof~\cite{cryptoeprint:2019/1076}. This enables the composition of multiple statements into a single recursive proof without relying on a trusted setup. In this sense, Plonky2 trades constant proof size for transparency and native support of recursion, making it well suited for applications where proofs must be aggregated or nested across several protocol layers~\cite{Plonky2Draft2022}.

\paragraph{Arithmetization and components} 
In the recursive implementation, the constraints are formulated with respect to the Plonky2 proof system. 
The relations defined by $C_{\text{cred-bind}}^{(1)}$--$C_{\text{cred-bind}}^{(4)}$ remain identical to those in the monolithic case (cf.~Section~\ref{sec:monolithic}) and are not repeated here. For the derived key binding, we introduce four additional constraints $C_{\text{key-bind}}^{(1)}$--$C_{\text{key-bind}}^{(4)}$, whose conjunction defines:
\[
R_{\text{key-bind}} = \{(x,w) : \bigwedge_{i=1}^4 f_{C_{\text{key-bind}}^{(i)}}(x,w)=0\}.
\]
In this equation, $f_{C_{\text{key-bind}}^{(i)}}(x,w)=0$ denotes a vector of polynomial constraints associated with the $i$-th constraint, and every polynomial in this vector must evaluate to zero. All elliptic-curve operations are defined over the prime field $\mathbb{F}_{p'}$ of secp256k1, while the recursive Plonky2 backend itself is executed over the Goldilocks field. To bridge this field mismatch, Plonky2 realizes the required non-native arithmetic via dedicated gates that operate on 32-bit chunks. This mechanism implicitly corresponds to a limb decomposition, with the splitting and range enforcement handled by the gate system rather than by explicit circuit constraints.

Constraint~$C_{\text{key-bind}}^{(1)}$ ensures consistency of the blockchain wallet parent public key across inner and outer proofs. The parent key $pk_0 \in E(\mathbb{F}_{p'})$ is provided as witness in the outer circuit and as public input in the inner proof. Equality is enforced limb-wise on the affine coordinates:
\[
f_{C_{\text{key-bind}}^{(1)}}(x,w) 
= \bigl(pk_{0,x}^{\mathrm{out}} - pk_{0,x}^{\mathrm{in}},\;
pk_{0,y}^{\mathrm{out}} - pk_{0,y}^{\mathrm{in}}\bigr) 
= (0,0).
\]
Here, $pk_0^{\mathrm{in}}$ denotes the value of $pk_0$ bound in the public input vector of $\pi_{\text{cred-bind}}$. Constraint~$C_{\text{cred-bind}}^{(2)}$ ensures consistency of the issuer public key $pk_I \in E(\mathbb{F}_{p})$, which appears as public input in both inner and outer circuits, and equality is enforced by:
\[
f_{C_{\text{key-bind}}^{(2)}}(x,w) 
= \bigl(pk_{I,x}^{\mathrm{out}} - pk_{I,x}^{\mathrm{in}},\;
pk_{I,y}^{\mathrm{out}} - pk_{I,y}^{\mathrm{in}}\bigr) 
= (0,0).
\]
More precisely, $pk_I^{\mathrm{in}}$ denotes the value of $pk_I$ bound in the public input vector of $\pi_{\text{cred-bind}}$. Constraint~$C_{\text{key-bind}}^{(3)}$ realizes the BIP32 non-hardened key derivation on secp256k1. Given the parent public key $pk_0 \in E(\mathbb{F}_{p'})$, chain code $cc_0$, and derivation index $i \in \{0,\dots,2^{31}-1\}$, the child key must satisfy:
\[
pk_i = pk_0 + [I_L]G.
\]
In this equation, $G$ is the generator of secp256k1 and $I_L$ is derived as specified in BIP32~\cite{bip32}. The corresponding arithmetization enforces the system:
\[
\begin{aligned}
	& M = \mathsf{serP}(pk_0) \,\|\, \mathsf{ser}_{32}(i) \;\wedge \\
	& I = \mathsf{HMAC\mbox{-}SHA512}(cc_0, M) = I_L \,\|\, I_R \;\wedge \\
	& 0 < I_L < n \;\wedge \\
	& (x_K,y_K) = [I_L]G \;\wedge \\
	& (x_i,y_i) = (x_0,y_0) + (x_K,y_K) \;\wedge \\
	& pk_i = (x_i,y_i). 
\end{aligned}
\]
Concretely, $\mathsf{serP}$ denotes SEC~1 compressed serialization, $I_L$ is parsed as integer from the left 256 bits of $I$, and scalar multiplication and point addition are decomposed into polynomial constraints via the Weierstraß addition law as in~$C_{\text{cred-bind}}^{(1)}$. Constraint~$C_{\text{key-bind}}^{(4)}$ realizes the recursive proof verification. It enforces validity of the inner proof $\pi_{\text{cred-bind}}$ with respect to the statement $(pk_I, pk_0)$ under the verification key $\mathsf{vk}_{\text{cred-bind}} \subseteq \mathsf{pp}_{\text{cred-bind}}$:
\[
f_{C_{\text{cred-bind}}^{(4)}}(x,w) = \mathsf{Verify}_{\text{Plonky2}}(\mathsf{pp}_{\text{cred-bind}}, (pk_I, pk_0), \pi_{\text{cred-bind}}) - 1 = 0.
\]
Here, $\mathsf{Verify}_{\text{Plonky2}}$ denotes the system of polynomial constraints implementing the recursive verifier, including consistency checks of Merkle openings and low-degree proofs in the \acrshort{fri} protocol.

\medskip
Taken together, these four components yield a system of polynomial constraints that define the relation $R_{\text{key-bind}}$ and constitute the input to the Plonky2 compilation step. In Plonky2, this system is represented by a collection of Plonkish gates, which at circuit construction time expand into polynomial constraint equations over the Goldilocks field. The resulting constraint polynomials are aggregated and their low-degree property is certified through the \acrshort{fri} protocol, with commitments realized via Merkle trees to guarantee both algebraic correctness and oracle consistency within the recursive verifier.

\paragraph{Fixed-base scalar multiplication optimization}
Since the proving backend changes from Groth16 to Plonky2, the inner circuit $C_{\text{cred-bind}}$ was re-implemented, which required a re-arithmetization of constraints $C_{\text{cred-bind}}^{(1)}$–$C_{\text{cred-bind}}^{(4)}$ over the Goldilocks field while leaving their formal definitions unchanged. In this context, the scalar multiplication $[u_2]Q_I$ constitutes the dominant cost component of \acrshort{ecdsa} verification ($C_{\text{cred-bind}}^{(3)}$). As in the monolithic case, a variable-base realization in Plonky2 necessitates a bit-decomposed double-and-add traversal of the scalar\footnote{The implementation of this optimization is available in our fork of the \texttt{plonky2-ecdsa} library: \url{https://github.com/sommer-ph/plonky2-ecdsa}.}. This traversal is expressed through non-native, 32-bit limb arithmetic using \texttt{U32} gates, which introduces explicit carry propagation, range-enforcement constraints, and conditional additions, thereby substantially increasing circuit size. Empirically, this design resulted in an inner circuit of 262\,144 rows and proof generation times of $94.48$\,s for the complete credential–wallet binding proof, confirming that variable-base multiplication remains a prohibitive cost factor in the recursive setting. To enable a fair performance comparison and to mitigate prover-side costs, we applied the same fixed-base optimization in Plonky2 as in the monolithic case. The issuer’s public key $Q_I$ was embedded as a compile-time constant, which permits precomputation of windowed multiples:
\[
P_{i,v} = v \cdot 2^{si} Q_I, \qquad v \in \{0,\dots,2^s-1\}.
\]  
The precomputed values can then be selected in the circuit through constant-time \texttt{RandomAccessGate}s. For the recursive setting we chose a window size of $s=4$, yielding $t = \left\lceil \tfrac{\lambda}{s} \right\rceil = \left\lceil \tfrac{256}{4} \right\rceil = 64$ windows, where $\lambda=256$ reflects the bit length of the scalar values on P-256. In contrast, the monolithic implementation employed $s=8$ ($t=32$). This design decision reflects a system-specific trade-off. While smaller windows increase the number of additions ($63$ instead of $31$), they also shrink the lookup tables to $64 \cdot 16$ precomputed points. In Plonky2 this choice is advantageous, since multiplexer gates (\texttt{RandomAccessGate}) are comparatively costly relative to elliptic-curve additions, whereas in Groth16 the inverse relation held. Overall, the optimization eliminates the full sequence of $\lambda = 256$ doublings and $\tfrac{\lambda}{2} = 128$ conditional additions required in the variable-base algorithm, reducing the cost to at most $63$ additions and $64$ lookups. In addition, the security properties established for the monolithic case remain valid, since the precomputation depends only on the public constant $Q_I$, and range checks as well as curve-consistency constraints are enforced in the same manner. Table~\ref{tab:plonky2-fixed-base} reports the effect of the fixed-base optimization on the complete inner circuit ($C_{\text{cred-bind}}^{(1)}$--$C_{\text{cred-bind}}^{(4)}$) and the corresponding inner proof. Despite this optimization, the circuit size remained fixed at $262\,144$ rows, and proof generation decreased only moderately from $94.48$\,s to $85.05$\,s. The reason follows directly from the Plonky2 prover model in Section~\ref{sec:recursive-plonky2}. While fixed-base multiplication eliminates a substantial portion of scalar multiplication logic, the overall circuit complexity remains above the threshold for a smaller evaluation domain. In particular, the combination of in-circuit SHA-256 hashing over the credential input, base64url decoding of public key coordinates, the two secret-to-public-key derivations, and the residual arithmetic of \acrshort{ecdsa} verification sustains the row count. Since the prover workload is dominated by $\alpha_{\text{FFT}} \cdot N\log N$ and $\beta_{\text{Merkle}} \cdot N\log N$, with $N$ rounded to the next power of two, the global complexity remains tied to the same domain size. As a result, the optimization improves local arithmetic but yields only a limited end-to-end speed-up, while verification time and proof size stay essentially unchanged.

\begin{table}[t]
	\centering
	\begin{tabular}{lrr}
		\toprule
		& \textbf{Inner proof (var-base)} & \textbf{Inner proof (fixed-base)} \\
		\midrule
		Circuit build time      & 55.79\,s    & 48.19\,s    \\
		Circuit rows            & 262\,144    & 262\,144    \\
		Proof generation        & 94.48\,s    & 85.05\,s    \\
		Proof verification      & 0.01\,s     & 0.01\,s    \\
		Proof size              & 179.6\,kB   & 179.6\,kB \\
		\bottomrule
	\end{tabular}
	\caption{Impact of fixed-base optimization for \acrshort{ecdsa} verification on inner circuit and proof.}
	\label{tab:plonky2-fixed-base}
\end{table}

\paragraph{Fixed-shape HMAC-SHA512 optimization}
Having applied the same inner-circuit optimizations as in the Groth16 case, the Plonky2 instantiation reaches a comparable optimization level. The outer circuit $C_{\text{key-bind}}$ consists of four constraints. Two of them ($C_{\text{key-bind}}^{(1)}$, $C_{\text{key-bind}}^{(2)}$) are simple key-consistency checks with negligible cost, and $C_{\text{key-bind}}^{(4)}$ performs recursive verification of $\pi_{\text{cred-bind}}$, whose efficiency depends only on the inner proof. The remaining constraint, $C_{\text{key-bind}}^{(3)}$, realizes the BIP32 non-hardened key derivation, which we arithmetize as follows:
\[
I \gets \mathsf{HMAC\mbox{-}SHA512}\bigl(cc_0,\, M\bigr) = I_L \,\|\, I_R,\qquad 
M = \mathsf{serP}\bigl(pk_0\bigr)\,\|\, \mathsf{ser}_{32}(i).
\]
Furthermore, we enforce:
\[
0 < I_L < n,\qquad pk_i = pk_0 + [I_L]G.
\]
Here, $G$ is the generator of \emph{secp256k1} and $n$ its group order. According to the BIP32 specification, the chain code has fixed length $|cc_0| = 32\ \text{bytes}$, and the message has fixed length $|M| = 37\ \text{bytes}$~\cite{bip32}. With the fixed input sizes from BIP32, the HMAC evaluation reduces to a straight-line circuit without data-dependent branching. Concretely, the inner hash processes $|K'\oplus \mathsf{ipad}|+|M|=128+37=165$ bytes and the outer hash processes $|K'\oplus \mathsf{opad}|+|H_1|=128+64=192$ bytes, where $K' = cc_0 \,\|\, 0^{96}$ denotes the chain code padded with 96 zero bytes. In both cases the length exceeds one block but remains below two, so each expands to exactly two SHA-512 compression rounds. Overall, the evaluation consists of four fixed compression calls, executed in a static order. This fixed shape arises because $\mathsf{ipad}$ and $\mathsf{opad}$ are hardwired constants, $K'$ specializes to $cc_0$ extended with 96 zero bytes, the XORs $K'\oplus\mathsf{ipad}$ and $K'\oplus\mathsf{opad}$ reduce to direct wiring, and the SHA-512 padding is fully determined for both $M$ and $H_1$. The resulting constraints for $C_{\text{key-bind}}^{(3)}$ are:
\[
\begin{aligned}
	&\text{(i)}\quad H_1=\mathsf{SHA512}\bigl((K'\oplus\mathsf{ipad})\|M\bigr),\qquad
	\text{(ii)}\quad H=\mathsf{SHA512}\bigl((K'\oplus\mathsf{opad})\|H_1\bigr),\\
	&\text{(iii)}\quad H=I_L\|I_R,\ \ 0<I_L<n,\qquad
	\text{(iv)}\quad pk_i = pk_0 + [I_L]G.
\end{aligned}
\]

While the optimization simplifies the circuit structure, it does not change the asymptotic complexity of HMAC-SHA512. This can be expressed through a simple \emph{cost model}. Let $\alpha_{\mathrm{comp}}$ denote the gate cost of a SHA-512 compression block, and let $\alpha_{\mathrm{xor}}, \alpha_{\mathrm{pad}}$ capture the surrounding XOR and padding logic. The generic HMAC incurs:
\[
\mathrm{cost}_{\mathrm{gen}} \;\approx\; 4\,\alpha_{\mathrm{comp}} \;+\; \alpha_{\mathrm{xor}}^{\mathrm{var}} \;+\; \alpha_{\mathrm{pad}}^{\mathrm{var}}.
\]
In this equation, $\alpha_{\mathrm{xor}}^{\mathrm{var}}$ and $\alpha_{\mathrm{pad}}^{\mathrm{var}}$ account for data-dependent construction (ipad/opad, variable-length XOR, dynamic padding). The fixed-shape instantiation reduces these to constants:
\[
\mathrm{cost}_{\mathrm{fix}} \;\approx\; 4\,\alpha_{\mathrm{comp}} \;+\; \alpha_{\mathrm{xor}}^{\mathrm{const}} \;+\; \alpha_{\mathrm{pad}}^{\mathrm{const}}.
\]
In this expression, $\alpha_{\mathrm{xor}}^{\mathrm{const}},\alpha_{\mathrm{pad}}^{\mathrm{const}} \ll 
\alpha_{\mathrm{xor}}^{\mathrm{var}},\alpha_{\mathrm{pad}}^{\mathrm{var}}$ because the fixed-shape construction removes variable XOR and padding costs. Since the four SHA-512 compression blocks dominate ($\alpha_{\mathrm{comp}}$ large in non-native $64$-bit arithmetic), the asymptotic complexity remains unchanged: $\mathrm{cost}_{\mathrm{fix}}\approx \mathrm{cost}_{\mathrm{gen}}$. Empirically, the outer circuit occupied $262{,}144$ rows with a proving time of $91.91$\,s, as the reduced XOR/padding logic did not lower the domain size. As explained earlier in the description of the Plonky2 prover, prover complexity is dominated by \acrshort{fft} and Merkle-tree operations over a domain of size $N$ that is rounded up to the next power of two. As the gate count remained in the same power-of-two bucket, the optimization yields no measurable speed-up, despite simplifying the implementation and removing branches. The same rounding effect was already visible in the preceding fixed-base scalar multiplication optimization of the inner circuit (Table~\ref{tab:plonky2-fixed-base}), where the row count stayed at $262{,}144$ and only a modest end-to-end improvement was observed.

\medskip
\noindent\emph{Correctness and security.}
The specialization is semantics-preserving. It implements standard HMAC-SHA512 for the fixed input lengths, with RFC-compliant key normalization, domain separation via $\mathsf{ipad}/\mathsf{opad}$, and FIPS-compliant padding. The derivation constraints $0<I_L<n$ and $pk_i = pk_0 + [I_L]G$ remain unchanged. No secret-dependent control flow is introduced, as all tables and pads depend only on public constants.

\paragraph{Poseidon-based key derivation optimization}
In the previous optimization we observed that the \acrshort{bip32} non-hardened derivation step, when realized faithfully with HMAC-SHA512, constitutes a major performance bottleneck. The HMAC requires two invocations of SHA-512, i.e. four full compression blocks, which dominate the gate count in the outer circuit. Even after tailoring the HMAC to the fixed message and key lengths of \acrshort{bip32}, the heavy cost of the SHA-512 gadget remains. We therefore propose a SNARK-friendly alternative that preserves the additive update rule of \acrshort{bip32}, but replaces HMAC-SHA512 by a Poseidon-based pseudorandom function. Conceptually, the derivation continues to satisfy:
\[
pk_i \;=\; pk_0 + [t]\cdot G.
\]
Here, $pk_0$ is the parent public key, $G$ is the generator of $\mathrm{secp256k1}$, and $t$ is a pseudorandom scalar derived from the parent context $(pk_0, cc_0, i)$. This preserves the algebraic form of Constraint~$C_{\text{key-bind}}^{(3)}$, which likewise enforces $pk_i = pk_0 + [\cdot]G$, but differs in the definition of the derived scalar. While $C_{\text{key-bind}}^{(3)}$ defines $I_L$ as the left half of an HMAC-SHA512 output, we here obtain $t$ from a Poseidon hash modeled as a random oracle. This form of additive key derivation is the rule underlying \acrshort{bip32}, whose exact security in the random-oracle model has been analyzed in the literature~\cite{Das2021}. While \acrshort{bip32} instantiates the random oracle via HMAC-SHA512, our optimization replaces it by the SNARK-friendly Poseidon permutation, thereby reducing the circuit cost while maintaining the same security guarantees. 

\medskip
Formally, let $n$ denote the group order of $\mathrm{secp256k1}$. We define the derived scalar as
\[
t \;\equiv\; \mathsf{Poseidon}(pk_0, cc_0, i, \mathsf{DomSep}) \bmod n.
\]
In this equation, $pk_0 \in E(\mathbb{F}_{p'})$ is the parent public key in affine coordinates, $cc_0 \in \{0,1\}^{256}$ is the parent chain code, $i \in \{0,\dots,2^{31}-1\}$ is the non-hardened index, and $\mathsf{DomSep}$ is a fixed domain separator ensuring independence from other uses of Poseidon. Here $\mathsf{Poseidon}$ denotes the SNARK-friendly hash permutation over the base field, instantiated with a fixed number of rounds and a maximum distance separable matrix, which we model as a random oracle. The reduction ensures $0 < t < n$, and the child key $pk_i$ according to the additive update rule.

Table~\ref{tab:poseidon-comparison} summarizes the impact of replacing HMAC-SHA512 by Poseidon in constraint~$C_{\text{key-bind}}^{(3)}$. We report only the outer circuit, since the inner proof remains unaffected. Both proof generation and verification times include the recursive verification of the inner proof, i.e.\ they are not isolated measurements of the key derivation step alone. The substitution reduces the number of circuit rows by a factor of four from $262,144$ to $65,536$.  This drop in domain size directly translates into a reduction of proof generation time from $91.91$\,s to $18.39$\,s and a proof-size decrease of about 11\%. This confirms that the dominant cost of the HMAC construction stems from the four SHA-512 compression blocks, which are eliminated entirely by the SNARK-friendly Poseidon permutation.

\begin{table}[t]
	\centering
	\begin{tabular}{lrr}
		\toprule
		& \textbf{Outer proof (fixed-shape HMAC)} & \textbf{Outer proof (Poseidon-based)} \\
		\midrule
		Circuit rows        & 262\,144    & 65\,536 \\
		Circuit build time  & 55.90\,s    & 9.52\,s \\
		Proof generation    & 91.91\,s    & 18.39\,s \\
		Proof verification  & 0.04\,s     & 0.01\,s \\
		Proof size          & 179.7\,kB   & 159.2\,kB \\
		\bottomrule
	\end{tabular}
	\caption{Impact of Poseidon-based key derivation on outer circuit and proof.}
	\label{tab:poseidon-comparison}
\end{table}

\medskip
\noindent\emph{Correctness and security considerations.}
Under the random-oracle assumption for Poseidon over the base field, we obtain:
\[
t \;=\; \mathsf{Poseidon}(pk_0, cc_0, i, \mathsf{DomSep}) \bmod n.
\]
This value is computationally indistinguishable from a uniform scalar in $\{1,\dots,n-1\}$, up to negligible bias introduced by the modular reduction. Since the transcript length is fixed and spans at least 256 bits, the bias is negligible. Alternatively, rejection sampling could be employed if desired. Domain separation via $\mathsf{DomSep}$ guarantees that the key derivation function is isolated from other uses of Poseidon, preventing cross-protocol collisions or multi-use re-binding~\cite{cryptoeprint:2019/458}. The additive update rule is:
\[
pk_i \;=\; pk_0 + [t]\cdot G.
\]
It is preserved exactly as in constraint~$C_{\text{key-bind}}^{(3)}$, with only the pseudorandom function instantiation altered. Hence the reduction arguments for verified derivation carry over with $I_L$ replaced by $t$, relying on the soundness of the outer SNARK, the soundness of the embedded inner proof, and the functional correctness of the derivation interface. While proofs in the random-oracle model do not automatically translate to real-world instantiations, the heuristic use of Poseidon as a random oracle is well aligned with its design goals and cryptanalytic guarantees~\cite{cryptoeprint:1998/011}.

\paragraph{Performance analysis}
Table~\ref{tab:plonky2-base} reports baseline metrics for the recursive Plonky2 realization of our construction. \emph{Inner} refers to the credential–wallet binding proof and comprises constraints $C_{\text{cred-bind}}^{(1)}$–$C_{\text{cred-bind}}^{(4)}$ of the construction. The \emph{dynamic} variant instantiates $C_{\text{cred-bind}}^{(3)}$ (signature verification) with a variable-base scalar multiplication, whereas the \emph{static} variant replaces this by a fixed-base scalar multiplication on the issuer public key. \emph{Outer} refers to the derived key binding proof and comprises constraints $C_{\text{key-bind}}^{(1)}$–$C_{\text{key-bind}}^{(4)}$. It \emph{recursively verifies} an \emph{Inner} proof and enforces the additional derivation and binding logic.

Unlike Groth16, Plonky2 dispenses with a \acrshort{srs} and adopts a transparent, \acrshort{fri}-backed PLONK-style design. As detailed in the Plonky2 prover description, the prover’s dominant costs stem from \acrshort{fft}s over the evaluation domain and Merkle-based polynomial commitments, and recursive verification re-checks \acrshort{fri} queries and Merkle openings of the inner proof. Consequently, we focus on structural metrics (circuit rows, build time) and core proving metrics (proof generation, proof verification, proof size). All values are averaged over ten runs.

Moving from dynamic to static in the inner circuit leaves the domain size unchanged at $262{,}144$ rows, but reduces proof generation time moderately from $94.48$\,s to $85.05$\,s. The outer circuit is unaffected, since its size and proving time are dominated by the HMAC-based derivation logic and remain constant around $91.91$\,s. Verification is efficient in all cases (below $0.05$\,s), and proof sizes are stable at about $180$\,kB. These baselines serve as the reference point for the subsequent outer circuit optimizations.

\begin{table}[t]
	\centering
	\begin{tabular}{lrrrr}
		\toprule
		& \textbf{Inner (dyn.)} & \textbf{Outer (dyn.)} & \textbf{Inner (stat.)} & \textbf{Outer (stat.)} \\
		\midrule
		Circuit rows        & 262\,144  & 262\,144  & 262\,144  & 262\,144 \\
		Circuit build time  & 55.79\,s  & 55.90\,s  & 48.19\,s  & 55.90\,s \\
		Proof generation    & 94.48\,s  & 91.91\,s  & 85.05\,s  & 91.91\,s \\
		Proof verification  & 0.01\,s   & 0.04\,s   & 0.01\,s   & 0.04\,s \\
		Proof size          & 179.6\,kB & 179.7\,kB & 179.6\,kB & 179.7\,kB \\
		\bottomrule
	\end{tabular}
	\caption{Plonky2 metrics for inner (dyn./stat.) and outer proofs.}
	\label{tab:plonky2-base}
\end{table}

Table~\ref{tab:poseidon-comparison} reports performance metrics for the outer proof $\pi_{\text{key-bind}}$, which encapsulates constraints $C_{\text{key-bind}}^{(1)}$–$C_{\text{key-bind}}^{(4)}$ and recursively verifies an inner $\pi_{\text{cred-bind}}$ proof. Two optimization steps were evaluated. The \emph{fixed-shape} HMAC realization and the replacement of HMAC-SHA512 by Poseidon. 

The fixed-shape optimization statically constrains the message and key sizes of HMAC-SHA512 to the fixed parameters required by \acrshort{bip32}, thereby avoiding auxiliary padding logic and redundant conditional branches in the SHA-512 gadget. While this modification is conceptually meaningful, its effect on the circuit size is negligible. The SHA-512 compression function still dominates, and its arithmetization into $U32$-gates remains essentially unchanged. Moreover, since Plonky2 rounds the evaluation domain to the next power of two, minor savings in gate count do not translate into a smaller circuit size. Consequently, the metrics of the fixed-shape variant coincide with those of the unoptimized outer (inner static) baseline in Table~\ref{tab:plonky2-base}.

In contrast, replacing HMAC-SHA512 by a Poseidon-based pseudorandom function yields a significant performance improvement. The number of circuit rows decreases by a factor of four (from $262{,}144$ to $65{,}536$), which reduces the polynomial evaluation domain and the complexity of the prover’s low-degree extensions. Circuit build time drops from $55.90$\,s to $9.52$\,s, and proof generation improves from $91.91$\,s to $18.39$\,s, corresponding to a $5\times$ speedup. Verification cost remains efficient ($0.04$\,s vs.\ $0.01$\,s), and proof size decreases slightly from 179.7\,kB to 159.2\,kB.

All reported values already include the recursive verification of the inner proof and therefore represent the full prover and verifier costs of the composed construction. The results highlight that the Poseidon-based optimization substantially reduces prover complexity, while verifier performance remains essentially unaffected.

Table~\ref{tab:plonky2-modular} reports performance metrics for two modular recursion strategies in Plonky2, \emph{serial recursion} and \emph{parallel recursion}. In the serial recursion pipeline, the proof logic is split into successive circuits $C_{\text{cred-bind}}^{(1)}$, $C_{\text{cred-bind}}^{(2)}$, $C_{\text{cred-bind}}^{(3)}$, $C_{\text{cred-bind}}^{(4)}$, and a final stage covering $C_{\text{key-bind}}^{(3)}$--$C_{\text{key-bind}}^{(4)}$. Each stage verifies the proof of its predecessor and adds its own constraint logic. Unlike the construction of the derived key binding proof~($\pi_{\text{key-bind}}$), the public key consistency checks $C_{\text{key-bind}}^{(1)}$ and $C_{\text{key-bind}}^{(2)}$ are not included in the final stage. In a serial pipeline, the corresponding cross-proof inputs $(pk_I, pk_0)$ are not directly available at that point, and adding them would require explicitly re-exposing and forwarding these values across stages. Since $C_{\text{key-bind}}^{(1)}$ and $C_{\text{key-bind}}^{(2)}$ perform only lightweight consistency checks, they were omitted for simplicity. The final stage thus performs the public key derivation with Poseidon and verifies the preceding chain, so that the overall transcript still represents the complete construction, analogous in functionality to the one-step outer proof of Table~\ref{tab:poseidon-comparison}. In the parallel recursion pipeline, the credential-wallet binding proof $\pi_{\text{cred-bind}}$ is decomposed into four sub-circuits ($C_{\text{cred-bind}}^{(1)}$, $C_{\text{cred-bind}}^{(2)}$, $C_{\text{cred-bind}}^{(3)}$, $C_{\text{cred-bind}}^{(4)}$). Each of these circuits generates its own proof, analogous to the corresponding stages in the serial pipeline. A separate outer circuit then performs the public key derivation with Poseidon ($C_{\text{key-bind}}^{(3)}$) and, in addition, recursively verifies the four inner proofs from $C_{\text{cred-bind}}^{(1)}$, $C_{\text{cred-bind}}^{(2)}$, $C_{\text{cred-bind}}^{(3)}$, and $C_{\text{cred-bind}}^{(4)}$ in a single step. This design avoids the strictly sequential dependency chain of the serial recursion, since all four inner circuits can be proven independently before being aggregated by the outer verifier. For comparability with the serial recursion pipeline, the public key consistency checks $C_{\text{key-bind}}^{(1)}$ and $C_{\text{key-bind}}^{(2)}$ were also omitted in the parallel case, although their inclusion would in principle be possible. When comparing these pipelines, we consider the proof generation time of the \emph{final} recursive proof, since this is the artifact that must ultimately be verified or passed on. In the serial pipeline, the last stage required $18.29$\,s, while in the parallel pipeline the final outer proof required $18.30$\,s. Both values are close to the one-step recursion baseline, and since the public key consistency checks $C_{\text{key-bind}}^{(1)}$ and $C_{\text{key-bind}}^{(2)}$ were omitted in this configuration, we do not interpret this difference as a genuine efficiency gain.

Two structural effects explain why modular recursion does not provide consistent improvements over the one-step design.
\emph{First, domain sizing in Plonky2 is always rounded up to the next power of two.} Even if an intermediate circuit such as $C_{\text{cred-bind}}^{(1)}$, $C_{\text{cred-bind}}^{(2)}$ or $C_{\text{cred-bind}}^{(4)}$ contains fewer constraints than the outer, its degree is rounded to $2^{15}=32{,}768$, $2^{16}=65{,}536$, or $2^{17}=131{,}072$ rows. This effect neutralizes potential savings from decomposing the construction, since the polynomial commitment and \acrshort{fri} components operate on the padded domain size rather than the raw constraint count.
\emph{Second, modular recursion introduces cumulative overhead.} In the serial pipeline, each stage requires its own circuit build, witness assignment, and proof generation, in addition to recursively verifying the previous proof. In the parallel pipeline, although $C_{\text{cred-bind}}^{(1)}$--$C_{\text{cred-bind}}^{(4)}$ can in principle be executed concurrently, the outer circuit must still verify all inner proofs in one step. These effects explain why the total prover times in Table~\ref{tab:plonky2-modular} ($107.55$\,s for serial and $106.17$\,s for parallel) remain significantly higher than in the one-step recursion baseline. In summary, while both serial and parallel recursion offer clean modular decompositions of the construction, they do not improve efficiency in Plonky2. The mandatory power-of-two domain alignment and the overhead of recursive verification outweigh potential constraint savings. For practical deployments, the one-step recursive proof remains the preferable choice.

\begin{table}[t]
	\centering
	\begin{tabular}{lrrrrrr}
		\toprule
		& \textbf{$C_{\text{cred-bind}}^{(1)}$} & \textbf{$C_{\text{cred-bind}}^{(2)}$} & \textbf{$C_{\text{cred-bind}}^{(3)}$} & \textbf{$C_{\text{cred-bind}}^{(4)}$} & \textbf{$C_{\text{key-bind}}^{(3\text{--}4)}$} & \textbf{Total} \\
		\midrule
		\multicolumn{7}{l}{\emph{Serial recursion}} \\
		Circuit rows        & 65{,}536  & 65{,}536  & 131{,}072 & 65{,}536  & 65{,}536  & -- \\
		Circuit build time  & 8.50\,s   & 8.37\,s   & 20.54\,s  & 8.14\,s   & 8.73\,s   & 54.27\,s \\
		Proof generation    & 14.76\,s  & 12.74\,s  & 42.77\,s  & 18.94\,s  & 18.29\,s  & 107.55\,s \\
		Proof verification  & 0.01\,s   & 0.01\,s   & 0.01\,s   & 0.01\,s   & 0.01\,s   & 0.01\,s \\
		Proof size          & 162.6\,kB & 162.4\,kB & 169.6\,kB & 162.9\,kB & 162.9\,kB & $\approx$166.5\,kB \\
		\midrule
		\multicolumn{7}{l}{\emph{Parallel recursion}} \\
		Circuit rows        & 65{,}536  & 32{,}768  & 131{,}072 & 65{,}536  & 65{,}536  & -- \\
		Circuit build time  & 10.90\,s  & 4.99\,s   & 23.67\,s  & 10.67\,s  & 11.36\,s  & 61.59\,s \\
		Proof generation    & 18.18\,s  & 7.10\,s   & 45.49\,s  & 17.05\,s  & 18.30\,s  & 106.17\,s \\
		Proof verification  & --        & --        & --        & --        & 0.01\,s   & 0.01\,s \\
		Proof size          & 162.6\,kB & 155.5\,kB & 169.4\,kB & 162.6\,kB & 162.9\,kB & $\approx$166.5\,kB \\
		\bottomrule
	\end{tabular}
	\caption{Plonky2 metrics for serial and parallel recursion.}
	\label{tab:plonky2-modular}
\end{table}

\section{Exploratory recursive implementation (Nova)}
\label{sec:recursive-nova}
As a third backend, we considered Nova\footnote{\url{https://github.com/microsoft/Nova}}, a recursion framework based on \acrfull{ivc} and folding. Rather than distributing the verifier logic across explicit recursive circuits as in Plonky2, Nova maintains an accumulator state that is incrementally updated at each step~\cite{cryptoeprint:2021/370}. In our setting, we aimed to integrate the credential–wallet binding proof $\pi_{\text{cred-bind}}$ with constraints $C_{\text{cred-bind}}^{(1)}$--$C_{\text{cred-bind}}^{(4)}$ into Nova using the Nova Scotia\footnote{\url{https://github.com/nalinbhardwaj/Nova-Scotia}} adapter, and to realize it as an initial folding step. This section describes the Nova prover model, the details of this integration, and the resulting performance observations.

\paragraph{Nova prover}  
Nova realizes \acrshort{ivc} using a novel \emph{folding scheme} rather than conventional recursive SNARKs. Each step $i$ is expressed as a (relaxed) \acrshort{r1cs} instance with witness $w_i$. The prover produces a proof that simultaneously attests to the correctness of this step and to the validity of an accumulator state folded from the previous step. Rather than re-executing verifier checks from scratch, they are aggregated into a compact algebraic accumulator that persists across steps. As a consequence, proof size and verifier runtime are independent of the recursion depth (i.e., $O(1)$ in $k$), and can be further compressed to $O(\log |F|)$ using a tailored \acrshort{zksnark}. Nova can be instantiated efficiently on elliptic curve cycles, such as the Pasta cycle (Pallas/Vesta), so that the verifier for one step can be expressed natively over the scalar field of the other curve in the cycle. On the prover side, each step’s cost is dominated by \acrshort{msm}s of size proportional to the circuit, with only a constant recursion overhead from folding. No \acrshort{fft}s are required, which makes Nova particularly efficient for long recursive computations~\cite{cryptoeprint:2021/370}. Heuristically, we model the prover’s cost for $k$ steps of identical size $N_0$ as:
\[
T_{\text{prover}}^{\text{Nova}}(k,N_0) \;\approx\; \beta_{\text{init}} \;+\; k\big(\alpha_{\text{R1CS}}N_0 + \gamma_{\text{fold}}\big).
\]  
Here, $\beta_{\text{init}}$ denotes implementation-specific initialization, $\alpha_{\text{\acrshort{r1cs}}} N_0$ accounts for the per-step proving work, and $\gamma_{\text{fold}}$ models the constant folding overhead. This heuristic formula is consistent with the asymptotic cost structure discussed in~\cite{cryptoeprint:2021/370}. Once $\beta_{\text{init}}$ is amortized, each additional step adds essentially the work of the step itself plus a constant folding overhead, while proof size and verifier runtime remain constant in $k$.  

\paragraph{Implementation and results} 
The constraints $C_{\text{cred-bind}}^{(1)}$--$C_{\text{cred-bind}}^{(4)}$ of the credential--wallet binding proof were arithmetized as described in Section~\ref{sec:monolithic} and compiled by Circom into an \acrshort{r1cs} instance. We integrated this \acrshort{r1cs} into Nova via Nova Scotia, which converts the circuit and witness into a relaxed \acrshort{r1cs} with commitment data. While this demonstrates that Circom circuits can in principle be expressed in Nova’s algebraic format, it does not enable true recursion. The crucial component, a verifier circuit for the prior step defined over the partner curve of the cycle, is absent. For example, while BN254 and Grumpkin form a cycle of elliptic curves suitable for Nova-style recursion, the Circom toolchain only produces an \acrshort{r1cs} over BN254 and does not generate a corresponding verifier circuit over Grumpkin~\cite{cryptoeprint:2023/573}. Hence, the recursive linkage between steps cannot be realized. Instead of folding the prior step into a compact accumulator, each step re-executes the full base relation. This prevents true recursion and causes subsequent steps to redundantly recompute the entire base circuit rather than merely updating an accumulator state. To obtain a full recursive composition in Nova, both the credential--wallet binding proof and the derived key binding proof would need to be re-implemented natively in Nova, together with gadget support for non-native elliptic-curve arithmetic. Since Nova does not currently provide support for the curves \texttt{secp256r1} and \texttt{secp256k1}, this engineering effort would have exceeded the scope of the present work. 
Nevertheless, the experiment is valuable as a contrast. While our prototype remained limited by tooling constraints, Nova’s folding paradigm illustrates how true recursion can, in principle, be achieved with only constant overhead per step. The asymptotic cost structure of Nova will therefore be included in the comparative cost model of Section~\ref{sec:analysis}, providing an important counterpoint to Groth16 and Plonky2.

\section{Performance and break-even analysis}
\label{sec:analysis}

\paragraph{Empirical observations}
Table~\ref{tab:consolidated-performance} summarizes the end-to-end performance of the optimized monolithic (Groth16), monolithic (Plonky2), and recursive (Plonky2) realizations. The table separates \emph{shared} metrics, which allow a direct head-to-head comparison, from \emph{system-specific} metrics that explain structural differences between the proving systems. The monolithic Groth16 proof is about $10\times$ faster to generate than the monolithic Plonky2 proof ($8.24\,s$ vs.\ $85.05\,s$) and more than two orders of magnitude smaller in size ($803\,B$ vs.\ $179.6\,kB$). Verification remains equally fast across all systems ($\approx 0.01\,s$), consistent with Groth16’s constant-size verifier and Plonky2’s succinct verifier design. In practical deployments, this translates into significantly lower prover latency and substantially reduced transmission or storage overhead for Groth16, while verifier-side costs remain negligible in all systems. Beyond these shared metrics, Groth16 entails one-time preprocessing costs for circuit compilation, witness generation, and the trusted setup (ranging from $65.93\,s$ to $504.42\,s$ in our measurements). Although these dominate the pipeline once per circuit, they amortize across multiple proofs. Plonky2, in contrast, avoids a \acrshort{srs} but incurs higher per-proof cost due to its polynomial \acrshort{iop} with \acrshort{fft}s, Merkle commitments, and \acrshort{fri} queries. The decisive driver of prover cost is the evaluation-domain size (\emph{rows}), which fixes the scale of these operations. Circuit build time adds further overhead ($48.19\,s$ monolithic, $9.52\,s$ recursive), yet the bulk of latency stems from proof generation itself ($85.05\,s$ monolithic, $18.39\,s$ recursive). The recursive Plonky2 realization with Poseidon-based derivation reduces proof generation time to $18.39\,s$ and proof size to $159.2\,kB$, substantially improving over monolithic Plonky2 but still falling short of the Groth16 baseline. While this confirms that Plonky2 is recursion-friendly in principle, the additional verifier gadget and domain-size padding prevent recursion from yielding advantages for a single instance of our construction. Nevertheless, when Plonky2 is considered in isolation, the recursive variant provides a clear efficiency gain over its monolithic counterpart, indicating that recursion can be beneficial within this proof system even if it does not yet outperform Groth16.

\medskip
\emph{Answer to the research question.}
For the credential–wallet binding proof $\pi_{\text{cred-bind}}$ considered in this work, the \emph{monolithic} Groth16 implementation achieves superior efficiency over the \emph{recursive} Plonky2 realization in both proving time and proof size, while verification cost remains negligible in both systems. In the Plonky2 case, recursion was instantiated through the derived key binding proof $\pi_{\text{key-bind}}$, which encapsulates $\pi_{\text{cred-bind}}$ as an inner proof and adds the corresponding derivation logic. The consolidated evaluation therefore answers the research question in favor of the monolithic approach for this use case. At the same time, it is important to consider Plonky2 in isolation. Here, the recursive realization with Poseidon-based key derivation clearly outperforms the monolithic variant, reducing proof generation time from $85.05\,s$ to $18.39\,s$ and proof size from $179.6\,kB$ to $163\,kB$. This indicates that, although Plonky2 recursion does not yet reach the efficiency of Groth16, it already provides tangible benefits over a monolithic Plonky2 instantiation. In the following section, we develop a formal cost model to explain the observed performance differences and to determine the parameter ranges in which recursion would become asymptotically competitive.

\begin{table}[ht]
	\centering
	\begin{tabular}{lccc}
		\toprule
		\textbf{Metric} & \textbf{Monolithic (Groth16)} & \textbf{Monolithic (Plonky2)} & \textbf{Recursive (Plonky2)} \\
		\midrule
		\multicolumn{4}{l}{\emph{Shared metrics}} \\
		\midrule
		Proof generation      & 8.24\,s     & 85.05\,s     & 18.39\,s \\
		Proof verification    & 0.01\,s     & 0.01\,s      & 0.01\,s \\
		Proof size            & 803\,B      & 179.6\,kB    & 159.2\,kB \\
		\midrule
		\multicolumn{4}{l}{\emph{System-specific metrics}} \\
		\midrule
		Non-linear constraints & 1\,320\,870 & --           & -- \\
		Linear constraints     & 429\,368    & --           & -- \\
		Circuit compilation    & 99.96\,s    & --           & -- \\
		Witness generation     & 65.93\,s    & --           & -- \\
		Trusted setup          & 504.42\,s   & --           & -- \\
		Key export             & 0.69\,s     & --           & -- \\
		Circuit rows           & --          & 262\,144     & 65\,536 \\
		Circuit build time     & --          & 48.19\,s     & 9.52\,s \\
		\bottomrule
	\end{tabular}
	\caption{Performance metrics of optimized implementations.}
	\label{tab:consolidated-performance}
\end{table}


\paragraph{Theoretical cost models}
The observed performance gap can be understood by placing the prover cost functions of Groth16, Plonky2, and Nova side by side:
\[
\begin{aligned}
	T_{\text{Groth16}}(N) &\;\approx\; \alpha_{\text{\acrshort{msm}}}\,N \;+\; \beta_{\text{FFT}}\,N\log N,\\
	T_{\text{Plonky2}}(N) &\;\approx\; \alpha_{\text{FFT}}\,N\log N \;+\; \beta_{\text{Merkle}}\,N\log N \;+\; \gamma_{\text{\acrshort{fri}}}\,N,\\
	T_{\text{Nova}}(k,N_0) &\;\approx\; \beta_{\text{init}} \;+\; k\big(\alpha_{\text{\acrshort{r1cs}}}N_0 \;+\; \gamma_{\text{fold}}\big).
\end{aligned}
\]

\emph{Comparative insight.}
Groth16 scales essentially linearly with comparatively small constants (batched \acrshort{msm} and efficient \acrshort{fft}s) and yields constant-size proofs, which matches the single, mid-sized circuit $\pi_{\text{cred-bind}}$ well. Plonky2, while transparent and recursion-friendly, carries larger constant factors from its \acrshort{fft}–Merkle–\acrshort{fri} pipeline and operates on an evaluation domain rounded to the next power of two, i.e., $N_{\mathrm{eff}}=2^{\lceil\log_2 N\rceil}$, where $N_{\mathrm{eff}}$ denotes the effective circuit size after power-of-two padding imposed by Plonky2’s domain construction. This padding fixes the amount of \acrshort{fft}, Merkle, and \acrshort{fri} work irrespective of small gate savings. In the recursive composition used here, the outer proof $\pi_{\text{key-bind}}$ embeds the inner $\pi_{\text{cred-bind}}$ and replays its \acrshort{fri} queries and Merkle openings via a verifier gadget, adding non-trivial per-instance overhead. Nova offers constant verifier cost and per-step linear proving with a constant folding term, making it asymptotically attractive for long pipelines, but current tooling and curve support prevents applying it to the secp256r1 or secp256k1 setting considered here. Together, these effects explain why Plonky2 is slower than Groth16 for a single mid-sized instance in our measurements and motivate the break-even analysis that follows.

\paragraph{Break-even analysis}  
An important and somewhat unexpected outcome of our evaluation is that, in a cross-prover comparison, recursion does not provide an advantage over monolithic Groth16 proofs. Intuitively, one might expect recursion to become favorable once individual circuits are moderately large, since recursive systems are often presented as a way to compress or amortize proving costs. In our setting, however, this intuition does not hold. When contrasting a \emph{one-step recursive} realization in Plonky2 with a \emph{monolithic Groth16} prover for the same statement size~$N$, the opposite effect is observed. Let $v(N)$ denote the size in gates of the Plonky2 verifier circuit that verifies a proof of size~$N$. According to the cost models introduced above, recursion would only be beneficial if:
\[
T_{\text{prover}}^{\text{Plonky2}}(N) \;+\;
T_{\text{prover}}^{\text{Plonky2}}\!\big(v(N)\big)
\;\le\;
T_{\text{prover}}^{\text{Groth16}}(N).
\]
This inequality reflects the cross-prover perspective adopted in our evaluation, where we ask whether a single recursive Plonky2 proof can replace a single monolithic Groth16 proof. If, by contrast, one were to compare functional equivalence (i.e., two Groth16 proofs vs. one recursive Plonky2 proof), the right-hand side would have to be scaled accordingly by a factor of two. Equivalently, define the advantage function as:
\[
\Delta(N)
:= T_{\text{prover}}^{\text{Groth16}}(N)
- T_{\text{prover}}^{\text{Plonky2}}(N)
- T_{\text{prover}}^{\text{Plonky2}}\!\big(v(N)\big).
\]
Therefore, recursion is beneficial iff $\Delta(N)\ge 0$. Two structural factors govern $\Delta(N)$. First, the size of the verifier circuit $v(N)$ is critical. The outer proof must re-execute Merkle-path checks and \acrshort{fri} query verifications of the inner proof, which substantially inflates $v(N)$. Unless the verifier is very small, this overhead quickly dominates. Second, the constant factors inherent in the Plonky2 pipeline matter significantly. The prover is dominated by hashing, Merkle construction, and \acrshort{fft}/\acrshort{fri} routines. Only substantial engineering improvements in these kernels would lower $T_{\text{prover}}^{\text{Plonky2}}(\cdot)$ enough to shift the balance in favor of recursion. At our evaluated instance $N=N_{\!\mathrm{exp}}$, we measured:
\[
T_{\text{prover}}^{\text{Groth16}}(N_{\!\mathrm{exp}})=8.24\,\text{s},\qquad
T_{\text{prover}}^{\text{Plonky2}}(N_{\!\mathrm{exp}})
+ T_{\text{prover}}^{\text{Plonky2}}\!\big(v(N_{\!\mathrm{exp}})\big)
= 18.39\,\text{s}.
\]
Hence $\Delta(N_{\!\mathrm{exp}})<0$. In other words, one-step recursion does not break even at this scale. Our experiments with serial and parallel recursion confirm this. Their final proof generation times ($\approx$18.30\,s) were still slower than the monolithic Groth16 baseline, so recursion offered no advantage in these variants either. For Plonky2, the verifier overhead $v(N)$ remains too large relative to the inner circuit, and padding effects exacerbate this gap. For Nova, the situation is conceptually different. Its folding scheme ensures that each additional step adds only linear cost in the step size plus a small constant for folding, while proof size and verifier runtime remain constant in the recursion depth. This means that Nova has a well-defined break-even regime. Once the number of steps $k$ is large enough, the accumulated Groth16 prover cost for $kN_0$ constraints will inevitably outgrow Nova’s linear growth. In other words, Nova becomes asymptotically favorable in long pipelines or iterative computations. In our case, the number of steps is too small to amortize the initialization term $\beta_{\text{init}}$, so the theoretical advantage of Nova does not manifest.

From a system-internal perspective, recursion in Plonky2 becomes beneficial whenever the cost of proving the verifier circuit $v(N)$ is smaller than the cost of generating another monolithic proof of size $N$. Formally, this requires:
\[
T_{\text{Plonky2}}(v(N)) \;\le\; T_{\text{Plonky2}}(N).
\]
In words, the recursive strategy is advantageous if the additional outer proof that re-verifies the inner proof incurs less cost than constructing a second monolithic proof from scratch. According to our measurements, this condition is clearly satisfied, since the verifier circuit is substantially smaller than the original statement circuit ($v(N)\ll N$). This explains why the recursive Plonky2 realization outperforms the monolithic Plonky2 baseline, even though it still falls short of Groth16 in a cross-prover comparison.

