\chapter{Conclusion}
\label{chap:conclusion}
% Summary
This thesis presented the design, implementation, and evaluation of a \acrshort{zksnark}-based system that bridges the gap between regulatory identity requirements in the EU and the privacy-preserving principles of decentralized Web3 architectures. The core contribution is the credential–wallet binding proof $\pi_{\text{cred-bind}}$, which enables a user to demonstrate possession of a valid EUDI credential, attesting identity, together with control over a blockchain wallet public key. On this basis, the derived key binding proof $\pi_{\text{key-bind}}$ was defined to attest that a child key used in a transaction is correctly derived from a parent key, while recursively verifying $\pi_{\text{cred-bind}}$ and thereby extending the identity binding transitively. Both constructions were analyzed and shown to satisfy completeness, knowledge soundness, and zero-knowledge, ensuring that the intended privacy and correctness guarantees are met under the assumptions of the underlying \acrshort{zksnark} system. The implementation and empirical evaluation compared the monolithic realization of $\pi_{\text{cred-bind}}$ with the recursive composition instantiated by $\pi_{\text{key-bind}}$. The results demonstrate that the monolithic construction achieves superior efficiency across proof size, generation time, and verification time, whereas the recursive approach introduces additional overhead without yielding practical benefits in the examined parameter regimes. From a cross-prover perspective, the monolithic Groth16 instantiation therefore constitutes the more practical solution for privacy-preserving and regulation-compliant credential–wallet integration in Web3 applications. At the same time, the evaluation also shows that, when focusing on Plonky2 alone, the recursive realization with Poseidon-based key derivation clearly outperforms its monolithic baseline. This indicates that, while Groth16 remains more efficient overall, recursion can already provide concrete benefits within Plonky2 itself.

% Interpretation
\medskip
The empirical results of this thesis show that the efficiency gap between monolithic and recursive composition is not primarily a matter of modularity, but of algebraic cost. Constraint counts are dominated by non-native arithmetic for secp256r1 and secp256k1, with \acrshort{ecdsa} verification as the main structural hotspot, and the SHA-256 over the credential header and payload as a second major contributor. In Plonky2, additional overhead arises from power-of-two domain padding and the embedded Merkle and \acrshort{fri} checks required for recursion. Fixed-base scalar multiplication and SNARK-friendly primitives such as Poseidon improve performance, but do not overturn the baseline advantage of a single Groth16 circuit at the studied scale. From an application perspective, this implies two things. First, privacy-preserving compliance for \acrshort{eudi}–Web3 integration is feasible with current tooling, yielding small proofs, and sub-second verification. Second, the current inefficiency of recursion should be seen as tool-dependent rather than inherent. While recursion remains attractive for aggregation, long pipelines, or transparent setups, the present verifier-inside-circuit overhead in Plonky2 prevents break-even at the explored scale when compared to Groth16. At the same time, our results confirm that recursion within Plonky2 itself already reduces proving time and proof size relative to its monolithic baseline, highlighting that the observed efficiency gap arises in comparison to Groth16 rather than within Plonky2. Finally, the findings suggest general design guidance. Immediate gains come from SNARK-friendly building blocks such as fixed-base elliptic curve operations and arithmetization-friendly hashes. Recursion provides benefits only once many statements are aggregated. For unlinkability, it is in practice simpler to regenerate a fresh credential–wallet binding proof for each public key rather than relying on derived key binding proofs, which avoids recursive overhead while preserving the intended privacy guarantees.

% Limitations
\medskip
The scope of this thesis is subject to several limitations. First, credential validation was restricted to signature verification. Attributes inherent to \acrshort{sdjwt} credentials, such as issuance time, expiry, and revocation status, were not incorporated into the circuit. Second, the construction assumes a single trusted issuer. In practice, an \acrshort{eudi} ecosystem involves multiple issuers across member states, which would require extending the model to a multi-issuer setting. This restriction directly stems from the applied optimization of fixing the issuer’s public key for efficient fixed-base scalar multiplications inside the circuit. Third, the recursive construction was restricted to a flat derivation setting, in which child keys are derived from the account-level key in a single non-hardened step. Full \acrshort{bip44} compatibility would require support for multi-level and hardened derivations, which are not captured in the present design. Fourth, the evaluation is tied to the chosen proving systems, namely Groth16 with RapidSnark and the recursive backend of Plonky2. Nova was additionally examined at a prototype level, but full integration was not achieved due to missing gadget support for the required curves. Finally, while the empirical performance demonstrates feasibility with proofs generated in a few seconds, further optimizations would be required to reach the efficiency targets of high-frequency or latency-critical applications.

% Future work
\medskip
Looking ahead, several directions emerge for extending this work. First, credential validation could be broadened beyond pure signature verification to cover additional attributes of \acrshort{eudi} credentials, such as issuance date, expiry, and revocation status. Integrating these checks would align the proof system more closely with production-grade requirements, at the cost of increased circuit complexity. Second, the present single-issuer assumption could be generalized to a multi-issuer setting. In the regulatory context of the EU, multiple trusted issuers per member state are expected. Supporting such a deployment would require dynamic handling of issuer public keys, for example by extending the fixed-base precomputation strategy with issuer-dependent lookup tables. Third, the recursive construction could be extended beyond the current flat derivation model to support full \acrshort{bip44} compatibility. This would enable multi-level and hardened derivations, allowing the proof system to cover the complete range of wallet hierarchies used in practice. At the same time, it should be noted that for the credential–wallet binding use case considered here, flat derivations or simply choosing a fresh key are entirely viable strategies. As a result, deep recursion layers are not required in practice, which further explains why the monolithic proof remains superior at the examined scale. Finally, further performance gains may be possible through refined circuit optimizations, as well as ongoing improvements in SNARK libraries and backends. Such advances hold potential to reduce constraint counts and proof generation time beyond the levels demonstrated here.
